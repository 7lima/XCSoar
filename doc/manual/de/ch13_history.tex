\chapter{Historie und Geschichte}\label{cha:history-development}
\section{Produkt Historie}

\xc startete als kommerzielles Produkt, welches von Mike Roberts (UK) entwickelt wurde. 
Das Programm hatte ein ein paar Jahre lang erfreulichen Erfolg und ging und durch mehrere Versionen.
Persönliche Gründe zwangen ihn, nach der  {\bf Version~2} die Weiterentwicklung einzustellen, sodaß
er sich entschied, den Code im späten Jahre 2004 der Öffentlichkeit zur Verfügung und  als
\xc {\bf Version~3} unter die  GNU Public License  zur Verfügung zu stellen.

Auf einer Yahoo Group wurde eine Webseite erstellt und es fanden sich einige Entwickler, die sich zusammentaten,
hier weiterzuentwickeln; damit war das Projekt geboren.

Im März 2005 war das Programm substantiell überarbeitet, was in der {\bf Version~4.0} endete.

Zu ungefähr dieser Zeit wurde die Koordination der einzelnen Mitentwickler immer schwieriger, der Programmkode gestaltete sich
schwieriger und zeitintensiver, so daß das Projekt auf SourceForge portiert wurde, wo mithilfe eines Versionsmanagement-Systemes
der Programmkode  verwaltet werden konnte.

Im Juli 2005 ist die  {\bf Version~4.2} herausgegeben worden, in welcher einige Probleme mit der Kompatibilität von PDAs bzgl.\
der GPS Hardwarekonfiguration behoben wurde.

Im September 2005 kam stand die {\bf Version~4.5} zur Verfügung. Hier wurden erstmal größere Änderungen bzgl.\ der Benutzeroberfläche
und -Eingaben vorgenommen; die Einführung  des internen ''Ereignis-Systemes'' und erste die Möglichkeit zur Bedienung der 
Oberfläche in anderen Sprachen   wurden vorgestellt.s

Im April 2006 ist die {\bf Version~4.7} für den \al vorgestellt worden - Stabilität und Performance standen hier an erster Stelle.
Viele Fehler wurden beseitigt, eine neue Methode für die Behandlung von Menüein- und ausgeben wurde erstellt,
welche aus XML-Dateien basierte.

Im September 2006 ist die  {\bf Version~5.0} für alle Plattformen (\al, PC, PDA) erschienen.
Diese Version zeichnete sich durch etliche Neuerungen aus und ist als erste massiv im Fluge und in 
Simulationen getestet  worden.

Die {\bf Version~5.1.2} kam im September 2007 für alle Plattformen (\xc, PC, PDA) heraus.
Auch hier sind etliche Neuerung vor der Herausgabe  durch intensive Test im Fluge und in Simulationen geprüft worden.
Die wichtigsten Neuerungen waren: 

Einführung des JPEG2000-Formates für die Karten, Einführung des \fl-Radars, Einführung der
OLC-Unterstützung, bessere Stabilität, Genauigkeit der Aufgenberechnung, bessere Bedienbarkeit.
Das erste mal haben hier wirklich viele Nutzer Eingaben und Vorschläge zur Verbesserung gemacht,
welche  zum Großteil berücksichtigt werden konnten.

Februar 2008:  {\bf Version~5.1.6} vorgestellt. Viele Fehler konnten beseitigt werden. Deutliche Verbesserung und
Erweiterung Einführung von AAT-Aufgaben sowie der RASP-Wettervorhersage.

März 2009, {\bf Version~5.2.2} ausgeliefert. Verbesserungen in der Bedienung und einige maßgeblich Verbesserungen sind
vorgenommen worden, unter anderem:

IGC-Files werden für die Validierung digital signiert z.B.\ für den OLC. Das erste Mal konnte das Programm auf Windows-CE 
und PNA-basierte Geräte portiert werden.

\fl ist in die Kartendarstellung integriert worden und es wird von nun an die \fl-Net Datenbank unterstützt.
Entwickler konnten von nun an \xc auch vom Linux-PC einfach kompilieren

August  2009: {\bf Version~5.2.4} interne Fehlerbehebung und Erweiterungen.

Im Dezember 2010 ist die {\bf Version~6.0} herausgegeben worden.
Nach einer intensiven, kompletten  Überarbeitung von großen Teilen des Programmkodes konnte viele Stabilitäts- und Performance
Probleme gelöst werden. Die Startzeit konnte dramatisch reduziert werden {(\small das kann ich als User nur bestätigen\dots Anmerkung OH)}

Viele neue Erweiterungen und Funktionen wurden eingeführt, inklusive eines neuen TaskManagers (Aufgabenverwaltung), 
erweiterter AAT-Fähigkeiten, einer neuen \fl-Unterstützung und einer Zentrierhilfe. 

Viele neue Sprachen standen zur Verfügung, und neue Übersetzungen konnten von nun an auch von
den Benutzern einfach vorgenommen werden.

Diese Umcodierung erlaubte \xc nun nativ auf Unix/Linux Systeme wie z.B.\ auch Android Geräten zu laufen
und das bei gleichzeitiger Performance-Steigerung durch moderne Compiler und aktuelle Hardware (z.B. Handys oder Tablets.)

März 2011: {\bf Version~6.0.7b} ist rausgekommen, die erste Version, die offiziell  Android-Geräte untersützt.

Januar 2013: {\bf Version 6.4.5.} Das erste komplette Handbuch auf Deutsch ist erschienen. (Danke, Hotte)

Februar 2013: \xc {\bf ist Weltmeister in der Clubklasse} in Argentinien! Mehr dazu auf unserer Hompage unter: \index{\xc ist Weltmeister!} \url{http://www.xcsoar.org/discover/2013/02/05/WGC_Argentina.html}

April 2013: {\bf Version 6.6} ist erschienen und bietet insbesondere erheblich besseren Bedienkomfort. Etliche GPS- Empfänger können  nun ausgewählt werden, die Bedienung ist erheblich intuitiver. 
Es gibt eine sog. ''cross section''-Ansicht, d.h. Lufträume werden nicht nur von oben, sondern auch seitlich zur Flugrichtung mit Gleitfluglinie analog der Polare angezeigt.


\section{Mach(t) mit!}\index{Mitmachen bei XCSoar!}

Der Erfolg diese Projektes ist das Resultat vieler einzelner Beitrage. Du mußt kein Software-Entwickler sein, um hier teilnehmen zu können.
Testberichte, Übersetzungen und Hinweise zu evtl.\ abnormalem Verhalten des Programme sind ebenso wichtig und gerne willkommen
im Projekt.

Generell gibt es fünf weitere Möglichkeiten, Projekt beizutragen, ohne Entwickler bzw. Programmierer zu sein:

\begin{description}
\item[{\bf 1. Gib uns Rückmeldungen}] Ideen, Vorschläge, Fehlerberichte und konstruktive Kritik sind gerne willkommen und werden
wo immer möglich berücksichtig.
\item[{\bf2. Hinweise zur Konfiguration}] Da \xc derart flexibel in der Konfiguration ist, brauchen wir immer Vorschläge von anderen, wie
das Programm am besten zu bedienen ist. Auch wir sind --wie jeder-- nicht immer gefeit vor einer gewissen ''Betriebsblindheit'' .
Vorschläge zur Bedienung von InfoBoxen, deren Layout, Anordnung, Farbe - alles was hilfreich ist um das Programm so sicher
und einfach wie möglich zu bedienen, bedürfen einiger Überlegung vorab, um diese ''Designstudien'' anschließend den Entwicklern
vorzustellen, welche es letztlich realisieren und als Standard im Programm implementieren.
\item[{\bf3. Integrität externer Daten}] Luftraumdateien und Wegpunktdateien sollten immer auf dem laufenden sein.
Hierzu sind oftmals Personen gefragt, die Kenntnisse der Lokalen Gegebenheiten haben (exakte Koordinaten, evtl. aktuelle Photos von Flug- und Landeplätzen).
\item[{\bf4. Werbung}]  Je mehr Personen die Software benutzen, desto besser wird das Programm werden. Je mehr Leute uns Rückmeldungen geben, umso
besser kann das Program auf die allgemeinen Vorlieben angepasst werden und umso schneller werden evtl.\ auch versteckte Fehler erkannt.

Du kannst helfen, indem Du die ''Werbetrommel'' schlägst, bspw.\ indem Du in Deinem Verein einen kleinen Vortrag hältst, oder aber im Vereinsraum mit
dem Beamer Deinen letzten Flug in \xc vorstellst oder nachfliegen läßt, oder indem Du einfach einem Kumpel das Programm in drei Minuten auf 
dem Handy zum Spaß oder zur Probe installierst.
\item[{\bf5. Dokumentation}]  Normalerweise und vollkommen klar - das Handbuch ist grundsätzlich nicht aktuell. Du kannst helfen, hier zu übersetzen oder zu 
schreiben. Oder einfach ein paar übersichtliche Kärtchen evtl.\ mit den wichtigsten ''Handgriffen'' erstellen, die Du an Deine Vereinskameraden 
verteilst.
\end{description}


\section{Philisophie des quelloffenen Projekts (Open Source)}

Es gibt einige Vorteile, ein Programm wie \xc als Open Source Projekt anzulegen:

\begin{itemize}
\item Zuallererst: es handelt sich um kostenlose, freie Software, das heißt, jeder (nicht nur piloten) Pilot kann es sich installieren und probieren und testen,
ob es das ist, was er braucht. Jeder ist frei, sich das Programm auf den PC, MAC, Pinguin, Android, PDA, PNA, EFIS  oder was auch immer zu installieren
und jeder kann es nach Belieben weitergeben.

\item Du hast freien Zugang zum Programmquellcode, sodaß Du ihn in jedem anderen freien OpenSource Programm verwenden darfst.

\item Programmcode im Internet offen zu haben, bietet die Gewähr, daß viele Leute einen forschenden Blick auf den Code werfen können und Fehler schnell offenbar 
werden um so schnell und beseitigt werden zu können.

\item  Eine große Gruppe von Entwicklern steht zur Verfügung um Hilfe bei Fehlern zu geben schnell neue Funktionen auszuarbeiten

\item Open Source - Software, welche unter die GNU Public License fällt, kann nicht zu einem späteren Zeitpunkt zu ''Löhnware'' 
gemacht werden. Die Benutzung derartiger Software wird daher auf keinen Fall dazu führen, daß Du später in die Kostenfalle tappst.
Diese Software ist und bleibt frei, gratis, kostenlos -- aber nicht umsonst\dots.

\end{itemize}

Die ausführlichen, kompletten Bedingungen zur Lizensierung von \xc sind in Anhang~\ref{cha:gnu-general-public} in deutsch und in englisch abgedruckt.

Die Entwicklung von \xc ist, seit es als OpenSource ins Netz gestellt wurde, eine rein unentgeltliche Arbeit von ausschließlich Freiwilligen.
Das bedeutet nicht, daß evtl.\  einige individuelle Entwickler oder Organisationen  kommerzielle Unterstützung anbieten - oder werden.
Der Geist dieses Projektes jedoch läßt vermuten, daß auch diese  ''Mitstreiter'' sicher der Gemeinschaft
mit Ihrem Produkt oder Dienstleistung einen guten Dienst leisten werden.


\section{Entwicklungpsozeß}

Wir versuchen, neue Funktionen so schnell wie möglich zu implementieren. Es muß hierbei natürlich abgewogen werden, das komplette Konstrukt
aus Bediener- und interner Funktionalität nicht derart abrupt zu verändern, sodaß ein Nutzer, welcher z.B.\ ein Update installiert, sich nicht
mehr zurechtfindet und im schlimmsten Falle von \xc abwendet.

So war es als wir z.B.\ die neue Menüoberfläche in V4.5 eingeführt haben,  notwendig, mit der neuen Version  auch ein externes
File mit zu verteilen, indem die althergebrachten Funktionen in einer Art Übersetzungstabelle enthalten waren.

\xc kann, wenn es während des Fluges benutzt wird, als ''kritische'' Software  (mission-critical) angesehen werden, da es sich um ein Echtzeit-System handelt.
Es wird daher sehr großer Wert darauf gelegt, daß diese Software ausgiebig von den Entwicklern und  freiwilligen Piloten im Fluge und in der ''harten Realität'' geprüft wird,
bevor eine neue Version an die Allgemeinheit herausgeht.

Die Tests im Fluge und die Erfahrungsberichte sind ganz sicher die besten  Prüfungen für das Programm, aber es gibt auch eine ganze Reihe 
von Berichten von Testern, die mit \xc im Auto herumgefahren sind und bestimmte Szenarien durchgespielt haben.  Eine ganze Menge von 
Testberichten konnten durch Zusendung von  IGC-Files ausgewertet werden, wo sekundengenau das Verhalten von \xc nachgespielt werden konnte.

Natürlich tun wir alles, um einen Programmabsturz oder -hänger vermeiden, wenn dies jedoch während der Testphase geschieht hat die Beseitigung eines genau solchen 
Fehlers oberste Priorität vor der evtl.\  Neueinführung von gewünschten und geplanten neuen Funktionen.
Stabilität des Programmes ist oberstes Gebot.


Es gibt daher für die immer weiter fortschreitende und sich extrem stark durchsetzende Android-Version eine Crash-Report-Verfolgung,
auf welche die Crash Reports der Androids, welche z.B.\  unter
\begin{tabbing}
{\small\texttt{/mnt/SDCard/XCSoarData/crash/..}}\\
\qquad\qquad{\small\texttt{../crash-2013-01-05-10-24-09-3998.txt}}
\end{tabbing}
oder
\begin{tabbing}
{\small\texttt{/mnt/extSDCard/XCSoarData/crash/..}}\\
\qquad\qquad{\small\texttt{../crash-2013-01-05-10-24-09-3998.txt}}
\end{tabbing}
abgelegt werden - je nachdem wie das File heißt und wo sich das XCSoarData-Verzeichnis befindet--
zur Begutachtung hochgeladen werden können. (Der Filename enthält das Datum sowie eine codierte Nummer)


Die Software Entwickler halten alle Kontakt miteinander über dies SourceForge-Entwickler-mailing- Liste:
\begin{quote}
\url{xcsoar-devel@lists.sourceforge.net}
\end{quote}
Wir versuchen, die Aktivitäten und Arbeiten so zu koordinieren, daß doppelte Arbeit und Konflikte vermieden 
und eine echte Teamarbeit zustande kommt. \index{Mitmachen-Ansprechpartner}
Wenn Du mitmachen willst, sende einfach eine mail an einen der Software-Entwickler aus der Liste s.u.\

\section{Die Basis der Benutzer}

Wer benutzt \xc? Gute Frage - schwer zu beantworten. 

Da niemand für diese Software bezahlt, und die meisten Leute das Programm 
anonym herunterladen oder kopieren, ist es schwer nachzuvollziehen,  wie viele Leute es auch tatsächlich nutzen.

Die Statistiken der Homepage zeigen für die Jahre von Juni 2005 bis Juni 2006 ca.\ 20 Downloads pro Tag an, und für 2006 bis 2007 80 Downloads pro Tag.

Wenn man sich anschaut, wie viele Leute die verschiedenen Karten und Topologie  herunterladen, kann man erkennen, daß \xc in vielen Ländern auf nahezu jedem Kontinent benutzt wird.

\xc wird von vielen Benutzergruppen verwendet, es gibt Wettbewerbspiloten, genauso wie ''Couchflieger'',
welche es im Zusammenspiel mit Segelflugsimulatoren wie z.B.\ Condor verwenden, oder aber Anfänger und ''Lust''-Flieger.

In immer mehr Vereinen wird \xc eingesetzt, da es eine echte Multiplikatorfunktion für die Nachwuchsflieger besitzt  und vor allem
in allen Flugzeugen für geringste Kosten als vollwertige Navigationshilfe eingesetzt werden kann:


Ein alter Ipaq 38xx liegt bei Ebay bei ca. 15-40\euro{}. Dazu ein bißchen Kabel, ein Spannungswandler für 3,5\euro{} und eine Halterung 
und mit deutlich unter 100\euro{} ist man dabei.  
Einige Vereine bieten \xc - Einführungstage an, um den kompletten ''Trupp'' auf demselben Level zu halten.

Hier zahlt sich auf beste Weise aus, daß \xc eben auch auf dem PC im Simulationsmodus läuft und so problemlos auf einem Beamer Im Vereinsheim 
ausgestrahlt werden kann.


\section{Dank an}\label{sec:credits}

{\large\bf Software Entwickler:}
\begin{compactitem}
  \item Santiago Berca \url{santiberca@yahoo.com.ar}
\item Tobias Bieniek \url{tobias.bieniek@gmx.de}
\item Robin Birch \url{robinb@ruffnready.co.uk}
\item Damiano Bortolato \url{damiano@damib.net}
\item Rob Dunning \url{rob@raspberryridgesheepfarm.com}
\item Samuel Gisiger \url{samuel.gisiger@triadis.ch}
\item Jeff Goodenough \url{jeff@enborne.f2s.com}
\item Lars H \url{lars_hn@hotmail.com}
\item Alastair Harrison \url{aharrison@magic.force9.co.uk}
\item Olaf Hartmann \url{olaf.hartmann@s1998.tu-chemnitz.de}
\item Mirek Jezek \url{mjezek@ipplc.cz}
\item Max Kellermann \url{max@duempel.org}
\item Russell King \url{rmk@arm.linux.org.uk}
\item Gabor Liptak \url{liptakgabor@freemail.hu}
\item Tobias Lohner \url{tobias@lohner-net.de}
\item Christophe Mutricy \url{xtophe@chewa.net}
\item Scott Penrose \url{scottp@dd.com.au}
\item Andreas Pfaller \url{pfaller@gmail.com}
\item Mateusz Pusz \url{mateusz.pusz@gmail.com}
\item Florian Reuter \url{flo.reuter@web.de}
\item Mike Roberts 
\item Matthew Scutter \url{yellowplantain@gmail.com}
\item Winfried Simon \url{winfried.simon@googlemail.com}
\item Google Inc., incl. Tom Stepleton \url{stepleton@google.com}
\item Simon Taylor \url{simon.taylor.uk@gmail.com}
\item Matthew Turnbull \url{matthewt@talk21.com}
\item Paolo Ventafridda \url{coolwind@email.it}
\item James Ward \url{jamesward22@gmail.com}
\item John Wharington \url{jwharington@gmail.com}


\end{compactitem}

\vspace{1em}
{\large\bf Dokumentation:}
\begin{compactitem}
  \item Daniel Audier \url{osteocool@yahoo.fr}
\item Monika Brinkert \url{moni@sunpig.de}
\item Kevin Ford \url{ford@math.uiuc.edu}
\item Claus-W. Häbel \url{c-wh@online.de}
\item Stefan Murry \url{smurry@ao-inc.com}
\item Adrien Ott \url{adrien.ott@gmail.com}
\item Helmut J. Rohs \url{helmut.j.rohs@web.de}
\item Mauro H. M. Tamburini \url{maurotamburini@hotmail.com}
\item Wolfram Zirngibl \url{rueckwaertsflieger@wolframz.net}


\end{compactitem}

\vspace{1em}
{\large\bf Übersetzer:}
\begin{compactitem}
  \item Milan Havlik
\item*Zdenek Sebesta
\item Tobias Bieniek \url{tobias.bieniek@gmx.de}
\item Niklas Fischer \url{nf@nordthermik.de}
\item Peter Hanhart \url{peter.hanhart@schoensleben.ch}
\item Max Kellermann \url{max@duempel.org}
\item Helmut J. Rohs \url{helmut.j.rohs@web.de}
\item Philipp Wollschlegel \url{folken@kabelsalat.ch}
\item*Thomas Manousis
\item Miguel Valdiri Badillo \url{catastro1@tutopia.com}
\item Alexander Caldwell \url{alcald3000@yahoo.com}
\item Diego Guerrero \url{iccarod@hotmail.com}
\item*Hector Martin
\item Andres Miramontes \url{amiramon@gmail.com}
\item*Romaric Boucher
\item Sylvain Burger \url{sylvain.burger@wanadoo.fr}
\item*Dany Demarck
\item*Zoran Milicic
\item*Sasa Mihajlovic
\item Gabor Liptak \url{liptakgabor@freemail.hu}
\item*Kalman Rozsahegyi
\item*Enrico Girardi
\item*Lucas Marchesini
\item*Rick Boerma
\item Joop Gooden \url{joop.gooden@nlr.nl}
\item Hans van 't Spijker
\item Michal Jezierski \url{m.jezierski@finke.pl}
\item*Mateusz Pusz
\item Luke Szczepaniak \url{luke@silentflight.ca}
\item Mateusz Zakrzewski
\item*Tales Maschio
\item Luis Fernando Rigato Vasconcellos \url{fernando.rigato@gmail.com}
\item Monika Brinkert \url{moni@sunpig.de}
\item Nikolay Dikiy
\item Brtko Peter \url{p.brtko@facc.co.at}
\item Roman Stoklasa \url{rstoki@gmail.com}
\item*Aleksandar Cirkovic
\item*Patrick Pagden
\item 'zeugma'
\item Morten Jensen
\item Kostas Hellas \url{kostas.hellas@gmail.com}
\item Alexander Caldwell \url{alcald3000@yahoo.com}
\item Xavi Domingo \url{xavi@santmodest.net}
\item Arnaud Talon
\item Adrien Ott \url{adrien.ott@gmail.com}
\item Matthieu Gaulon
\item Filip Novkoski \url{f1novkoski@gmail.com}
\item Szombathelyi Zolt\'an \url{szombathelyi.zoltan@main.hu}
\item \'Ur Bal\'azs \url{urbalazs@gmail.com}
\item Paolo Pelloni \url{paolo@paolopelloni.it}
\item Piero Missa \url{pieromissa@virgilio.it}
\item Masahiro Mori \url{mron@n08.itscom.net}
\item Jinichi Nakazawa \url{jin-nakazawa@wkk.co.jp}
\item Mike Myungha Kuh
\item Rob Hazes
\item Thomas Amland \url{thomas.amland@gmail.com}
\item Quint Segers
\item Wil Crielaars \url{kawa1998@home.nl}
\item Krzysztof Kajda
\item Michał Tworek
\item Tiago Silva
\item Mario Souza
\item J\'ulio Cezar Santos Pires \url{juliocspires@gmail.com}
\item Wladimir Kummer de Paula
\item Pop Paul \url{poppali1@yahoo.com}
\item Dobrovolsky Ilya \url{ilya_42@inbox.ru}
\item Mats Larsson \url{mats.a.larsson@gmail.com}
%\item Мирзаева Асаль

\end{compactitem}

$\star$:  Beiträge zum  LK8000 Projekt (\url{http://www.lk8000.it/}) hinzugefügt

\vspace{1em}
{\large\bf Kode und Algorythmen wurden weiterhin integriert von:}
\begin{description}
  \item[Efêmeris] Jarmo Lammi
\item[Shapelib] Frank Warmerdam\\ \url{http://shapelib.maptools.org}
\item[Least squares] Curtis Olson\\ \url{http://www.flightgear.org/~curt}
\item[Formulário Aviônico] Ed Williams\\ \url{http://williams.best.vwh.net/avform.htm}
\item[JasPer] Michael D. Adams\\ \url{http://www.ece.uvic.ca/~mdadams/jasper/}
\item[Apoio ao Volkslogger] Garrecht Ingenieurgesellschaft
\item[Analisador de vento girando] Andr\'e Somers\\ \url{http://www.kflog.org/cumulus/}.


\end{description}
