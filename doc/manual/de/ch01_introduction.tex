\chapter{Einführung}\label{cha:introduction}
Dies ist das \textsf{XCSoar} - Handbuch, geschrieben für \textsf{XCSoar} - Anwender. 


\textsf{XCSoar} ist ein Segelflugrechner-Programm, welches ursprünglich einmal für den Segelflugrechner \al der Firma Triadis geschrieben wurde.  
Anschließend waren die vor ca. 10 Jahren aufgekommenen PocketPC das Ziel, in der Zwischenzeit läuft \textsf{XCSoar} jedoch auf fast allen Plattformen und ist derzeit auf Android mit der derzeit 
sehr performanten Hardware zum Laufen gebracht worden. 

Diese Hardware eignet sich ideal, da die allermeisten der Geräte zwischenzeitlich über GPS- und Lagesensoren und einen 
meist recht fixen Prozessor mit entsprechend RAM verfügen, von denen die PDA damals nur zu träumen wagten - und dennoch 
läuft das Program auch dort hervorragend. 


Es wird hier vorausgesetzt, daß der Leser über Grundbegriffe des Fliegens insbesondere des Überlandfluges 
und über die MC-Theorie verfügt, denn ein Lehrbuch für das Fliegen an sich ist das hier nicht.


Es werden regelmäßig Updates herausgebracht, dem Benutzer ist empfohlen, hin und wieder auf der offiziellen Homepage nach zuschauen. 

Man sollte sich die ''release notes'', also die Veröffentlichungen zu den jeweiligen Versionen ansehen, um  auf evtl.\ Änderungen innerhalb des Programmes gefasst zu sein. 
Die komplette Dokumentation ist erhältlich auf  
\begin{center}
\url{http://www.xcsoar.org}
\end{center}
Da \textsf{XCSoar} ein internationales, offenes Projekt ist (an dem jeder mitmachen kann), sind sogut wie alle Vorgänge und Hinweise auf der Homepage auf Englisch. 
Es ist jedoch inzwischen auch ein deutsches Forum gegründet worden, auf dem man sich Informationen holen kann und mit anderen Anwendern über das ein oder anderen auch über Hardware diskutieren kann.
\begin{center}
\url{http://forum.xcsoar.org/viewforum.php?f=14}
\end{center}
oder aber unten auf $<$deutsch$>$ klicken

\section{Einteilung diese Handbuches}
Das Handbuch ist geschrieben, um den Piloten die bestmögliche Unterstützung bzw.\ Einleitung in das Programm zu geben. 
Wir haben uns bemüht, dies vor allem aus Pilotensicht darzustellen (und hoffen, daß das gelungen ist) 

Dies Kapitel beschäftigt sich vor allem mit dem Herunterladen und Installieren des Programmes. 
In Kapitel~\ref{cha:interface} wird auf das Bedienungskonzept eingegangen und soll einen Überblick über das Aussehen des Programmes geben.

Das Kapitel~\ref{cha:navigation} beschreibt die Kartendarstellung ''Moving Map'' im einzelnen  und zeigt, wie das Programm allgemein 
zur groben Unterstützung der benutzt werden kann. Kapitel~\ref{cha:tasks} zeigt auf, wie Überlandflüge geplant, eingegeben, geflogen und gespeichert werden, und bietet eine Übersicht
der Werkzeuge zur Analyse des aktuellen Fluges, um dem Piloten zu helfen,  seine Performance zu steigern.
Im Kapitel~\ref{cha:glide} geht es weiter ins Detail des Segelflugrechners. 
Es ist wichtig, sich mit diesen Details vertraut zu machen, da hier bestimmte Routinen des Programmes erläutert werden.
Externe Sensoren angeschlossener Geräte, Wetter und Atmosphäre werden in Kapitel\ref{cha:atmosph} behandelt, um zu zeigen, wie das Programm 
sich Konvektionsvorhersage, Wind und Thermik in eine \textsf{XCSoar}-interne Wettervorhersage und Flugplanung einbinden läßt.  
Im Kapitel~\ref{cha:airspace} wird beschrieben, wie \textsf{XCSoar} mit Lufträumen umgeht, davor warnt und diese darstellt. 
Weiterhin werden Kollisionswarnungen, welche durch das \fl-System ermittelt werden können behandelt.
Unter dem Abschnitt~\ref{cha:avionics-airframe} wird aufgezeigt, wie sich \textsf{XCSoar} mit diversen anderen Geräten und Systemen 
verbinden läßt z.B.\ auch der Steuerung über externe Schalter wie z.B.\ Wölbklappenschaltern. 

Der Rest des Handbuches beschäftigt sich hauptsächlich mit Referenzmaterial:

Kap.~\ref{cha:infobox} beschreibt die Funktion und den Inhalt der vorhandenen Infoboxen, die auf dem Bildschirm von \textsf{XCSoar} ausgewählt und angezeigt werden können.
Die Konfiguration des Programmes wird in Kap.~\ref{cha:configuration} ausführlich beschrieben. 
Die Formate der einzelnen Files und Daten, welche von \textsf{XCSoar} benutzt, im- und exportiert werden und wo diese zu finden sind, 
wird in Kap.~\ref{cha:data-files} aufgeführt. 

Zum Schluß soll eine kurzer Überblick über die Historie und den Entwicklungsgang von \textsf{XCSoar} zeigen, wie das Programm zu dem wurde, was 
es derzeit ist\dots Bei Interesse: Kap.~\ref{cha:history-development}.

\section{Notes}


\subsection*{Terminologie}
Im Handbuch werden einige Abkürzungen benutzt, welche evtl.\ --aber hoffentlich nicht--  erklärungsbedürftig sind.

So  steht z.B.\  PDA oder ''Organizer'' für eine Reihe von Geräten, welche es mittlerweile schon kaum noch 
gibt, da die Android-Geräte den Markt unglaublich schnell überschwemmen\dots  
{\small Lediglich eine Clique von Segelfliegern deckt sich mit den alten Ipaq ein, da diese für derzeit 15-25Euro in der Bucht 
zu ersteigern und absolut geeignet sind, um mit \textsf{XCSoar} zu funktionieren}

PDA steht für ''Portable Digital Assistant'' (mobiler digitaler Assistent auf rau-Deutsch). 
Ein PNA ist ein ''personal navigation assistant'' also ein ''privater Navigations Assistent ein NAVI!!

Naja\dots Auch die ''organizer'' gehören dazu - also ''Aufräumer'' oder ''Organisierer'' 
 
Egal, und wie auch immer, innerhalb diese Handbuches sollen diese Bezeichnungen lediglich dafür stehen, 
auf welchen Plattformen und Betriebssystemen \textsf{XCSoar} läuft und eine Gruppe von Anzeigeinstrumenten bezeichnen. 

\textsf{XCSoar} gibt es auch für den Triadis \al - Segelflugrechner. 
Für diesen Rechner wurde \textsf{XCSoar} ursprünglich einmal programmiert. Alles, was in diesem Dokument beschrieben wird, wird also auch 
mit dem \al funktionieren (die andere Bedienweise ausgenommen, da der \al kein Touchscreen ist - ebenso wie PC.)

\subsection*{Bildschirmphotos (Screenshots)}

Im Handbuch werden einige Screenshots dargestellt, um die Bedienung/Funktionalität eindrucksvoller und einfacher darstellen zu können.
Natürlich werden sich diese Screenshots von Gerät zu Gerät unterscheiden, wir gehen aber davon aus, daß es nicht sooo 
schwierig sein wird, eine quer-Darstellung  von der Hochkant-Darstellung zu unterscheiden und hoffen, daß im Zweifelsfalle der geschriebene
Text zur Klärung  beitragen wird. 

Es wird daher mitunter Verwirrung geben, da viele der hier aufgenommenen Bilder in der Hochkant-Ansicht aufgenommen wurden 
--wir können nicht für jedes einzelne Gerät entsprechende Screenshots aufnehmen, das würde dies Handbuch auf 400 Seiten aufblähen! 

Wie gesagt, wir versuchen, diese Photos up2date zu halten, es kann aber auch sein, daß mal ein altes bilde auftaucht.

Im Zweifel gilt der geschriebene Text!!!

\section{Plattformen}
\begin{description}
\item[Windows PC]
\textsf{XCSoar} läuft problemlos auf Windows.

Wenn ein GPS angeschlossen und für die Ausgabe der GPS-Daten konfiguriert wurde, kann vollkommen problemlos mit \textsf{XCSoar} gearbeitet 
werden.  (z.B.\ unterwegs auf einem Notebook o.ä.)

Der Simulator läuft ebenfalls vollkommen problemlos, falls keine GPS-Quelle angeschlossen ist!

Jeder, der mit Windows arbeitet und Segelflieger ist (und \textsf{XCSoar} benutzt oder es beabsichtigt), sollte unbedingt die 
PC-Version installieren und am Simulator trainieren!!! 
\item[Windows Mobile PDA/PNA]
PDA und PNA mit Windows mit Microsoft Pocket PC 2000 bis hinzu Windows Mobile 6 laufen mit \textsf{XCSoar} - der Autor hat einen Ipaq3850 seit Jahren 
auch auf Wettbewerben im Einsatz. Es geht!!

Windows Mobile 7 ist nicht unterstützt, da Microsoft entschieden hat, keine Unterstützung für Entwickler dieses 
Betriebssystemes zu geben.
\item[Unix/Linux PC]
\textsf{XCSoar} kann problemlos am WINE-Emulator laufen, es läuft aber auch (seit V6.0x) nativ für die Debian/UBUNTU) Versionen.
\item[Android Devices] \textsf{XCSoar} läuft auf Android 1.6 oder neuer 
\item[\al] Der \al ist ein Segelflugrechner welcher fabrikmäßig mit \textsf{XCSoar} als Software ausgestattet ist. 
Die \al-PRO-Version enthält ein integriertes GPS.
\end{description}


\section{Technische Unterstützung / Support}

\subsection*{Troubleshooting}
Ein kleines Team von freiwilligen, nicht bezahlten Leuten entwickelt und programmiert \textsf{XCSoar}. 

Obwohl wir gerne helfen, können wir nicht jede Frage beantworten, welche sich z.B. mit einem 
Hardware Fehler ''Mein  Galaxy SII bootet nicht mehr'' beschäftigt. Wir denken, das ist unmittelbar einleuchtend.

Es gilt, Software von Hardware zu trennen. 

Wenn Du Fragen zu \textsf{XCSoar} hast, sende eine --konkrete-- mail an: 

\begin{quote}
\url{xcsoar-user@lists.sourceforge.net}
\end{quote}
oder stelle deine Frage auf dem User Forum:
\begin{center}
\url{http://forum.xcsoar.org/}
\end{center}

Alle häufig auftretenden Fragen werden gesammelt und zur Website weitergeleitet, wo sie behandelt werden.
Du kannst an die die mailing list schreiben, um aktiv informiert zu werden, sowie Neuigkeiten stattfinden: 

Mehr dazu auf der \textsf{XCSoar}-Homepage. 

\begin{quote}
\textsf{XCSoar}soarwebsite
\end{quote}

Das beim Start von \textsf{XCSoar} erzeugte log-file \verb|xcsoar-startup.log| ist hilfreich für die Analyse von Fehlern und kann  
mit einem kleinen Bericht zu den Entwicklern gesendet werden, um Fehlern auf die Spur zu kommen 
und \textsf{XCSoar} permanent zu verbessern. 

Für \al Benutzer: dies Files wird vom \al auf das ''FromAltair'' Verzeichnis geschrieben, von wo es aus kopiert 
und eingesendet werden kann. 

\subsection*{Updates}
Du solltest periodisch nachschauen, ob Neuerungen/neue Versionen  bzgl.\ \textsf{XCSoar} auf der HP zu finden sind. 
Die Installation ist immer die gleiche; alle vom Bediener erstellten Files wie Konfiguratiobnsfiles 
bleiben beim Update bestehen und können später wieder benutzt werden. Die Updates auf der Android Plattform können -- je nach Einstellungen auf Deiner Hardware  (automatische updates erlaubt\dots)-- vollkommen automatisch erfolgen! 

Nach wichtigen Files wie Karten, Lufträumen und Wegpunkten zu schauen ist wohl selbstverständlich!


Wie jede komplexe Software kann auch \textsf{XCSoar} Fehler enthalten und ist \textsf{XCSoar} darauf angewiesen, daß diese dem Entwicklerteam zur  Kontrolle/Korrektur zugehen. Keine Rückmeldung--keine Verbesserung!

Hierzu gibt es extra ein Bug-tracking-System, also ein Fehler-Verfolgungssystem, in dem Ihr explizit Fehler melden könnt.

Bitte seht vorher nach, ob nicht ein solcher Fehler bereits gemeldet wurde. 
Doubletten machen unglaublich viel Arbeit und behindern den zügigen Fortgang der Entwicklung --logischerweise.  


Hierzu siehe:  
\begin{quote}
\url{http://www.xcsoar.org/trac/}
\end{quote}
oder einfach eine mail an 
\begin{quote}
\url{xcsoar-devel@lists.sourceforge.net}
\end{quote} 

\subsection*{Updaten von \textsf{XCSoar} auf dem \al}

Ein Update von \textsf{XCSoar} auf dem \al  benötigt als erstes das aktuelle  File {\tt XCSoarAltair-YYY-CRCXX.exe}, 
heruntergeladen z.b.\ auf einen USB-Stick. 

Anschließend wird der Sticke eingeschoben, der \al gestartet und es öffnet sich \al-eigene Bootsoftware, 
innerhalb welcher ein Update angewählt werden kann.

Für Details, schaut bitzte im \al-Handbuch, dem {\em Altair Owner's Manual} nach.
Andere Dateien wie Lufträume, Karten etc. werden auf die gleiche Weise auf den \al kopiert. 

\section{Training}
Es ist grundsätzlich allen Piloten,  insbesondere jedoch denen, die einen Segelflugrechner,  egal welcher Art, an Bord haben, 
dringenst angeraten, aus dem, Fenster zu schauen, anstatt sich ''{\it mit dere olle Gnöbbsche zu befasse\dots}''
\achtung 
Das gilt auch und selbstverständlich für diejenigen, welche \textsf{XCSoar} benutzen. \textsf{XCSoar} fliegt {\sl nicht} allein und hält {\sl nicht} 
die Fahrt und weicht {\sl keinem}  anderen Flugzeug oder Berg aus! 

Aus diesem Grunde ist die Beschäftigung bzw.\ Training mit \textsf{XCSoar} dringend angeraten, {\sl bevor} man es benutzt. 


\subsection*{\textsf{XCSoar} auf dem PC benutzen}
Die PC-Version sollte benutzt werden, um es den meisten (Windows)-Usern es zu erlauben, sich mit \textsf{XCSoar} ein 
bißchen auseinanderzusetzen. 

Da die Konfiguration exakt gleich ist, wie auf den mobilen Geräten, sollte es anschließend keine Schwierigkeit geben,
auch mit dem Handy oder PDA mit \textsf{XCSoar} 'abzuheben''.


Die PC-Version kann ebenfalls an externe GPS-Quellen angeschlossen werden und zu einem 
vollwertigen Navigationssystem ''verwandelt'' werden. 


\begin{itemize}
\item Schließe z.B.\ ein \fl an den PC an und benutze diesen als Bodenstation für \fl-Verkehr
\item Schließe den PC an ein intelligentes Variometer an und teste die die Konfiguration damit  (z.B. Vega)
\end{itemize}

\subsection*{\textsf{XCSoar} zusammen mit einem Flugsimulator benutzen und erlernen}
Es ist eine wirklich außerordentlich gute Idee und bringt echt was, \textsf{XCSoar} zusammen mit einem Flugsimulator zu benutzen, bzw.\ zu erlernen.
Dazu muß der Simulator/der PC die Daten mittels des Standard-NMEA-Protokolles ausgeben können, sodaß  \textsf{XCSoar} diese über den seriellen Port auslesen und interpretieren kann.
{\sc Condor} und {\sc X-Plane} bieten derzeit solche Möglichkeiten.  

Der große Nutzen des Trainings am Boden ist der, daß Du in der Luft die Griffe, Interpretationen der Anzeige und Möglichkeiten, 
die Dir \textsf{XCSoar} bietet ''im Schlafe'' beherrscht. Fliegst Du Wettbewerb, so ist das zumindest im Einsitzer absolut unumgänglich.
Auch ein LX8000 ist nicht innerhalb einer halben Stunde (auch nicht innerhalb einer ganzen) zu beherrschen -- und ich weiß wovon ich spreche\dots


\section{\textsf{XCSoar} sicher und verantwortungsbewußt benutzen}
Die Anwendung eines interaktiven Systemes wie  \textsf{XCSoar}, welches sogar Luftverkehrserkennung bietet, läßt den Benutzer dazu neigen, die Luftraumbeobachtung evtl.\ zu vernächlässigen. 
Es ist nicht unsere Intention, Piloten von der Beobachtung des Luftraumes um sie herum abzulenken,  sondern Ihnen  ein Hilfsmittel an die Hand zu geben,
mit dem u.a.\ die Navigation  und die Berechnung  des Endanfluges vereinfacht wird.

Die Philosophie, die hinter der Entwicklung dieser Software steht, ist die, die Ablenkung des Piloten von der Beobachtung wirklich wichtiger Dinge
wie z.B.\ (und vor allem) des in unmittelbarer Nähe befindlichen Luftraumes so gering wie möglich zu halten.

Hierzu ist vor allem Wert darauf gelegt worden, die Bedienereingaben auf ein Minimum --und wenn, dann so einfach und 
schnell wie möglich-- zu beschränken. 
 
Piloten, die  \textsf{XCSoar} benutzen, sollten sich der Verantwortung durch die evtl.\ Ablenkung durch ein solches Programm/System stets bewußt sein!

Es sollte selbstverständlich sein, folgende Dinge zu beachten:
\begin{itemize}
\item Werdet vertraut mit \textsf{XCSoar}, z.B.\ und insbesondere durch die Bedienung des Programmes z.B.\ mit Hilfe des Simulators
\item Benutze und ''fummele'' an Deinem Gerät erst dann herum, wenn Du {\sl sicher bist, daß der Luftraum {\bf frei} ist}!! 
Mache keine Kurven oder Turns, während Du mit der Bedienung der Software beschäftigt bist!An \textsf{XCSoar} herumspielen kannst Du am Boden. 
Du solltest mit der Bedienung wirklich fit sein, gerade auch im Wettbewerb, oder als ''Beginner''.
\item Konfiguriere Dein System so (es ist möglich und Du kannst es\dots), daß Du die interaktiven Eingaben auf ein Minimum beschränkst und 
jederzeit im Griff hast. (Es werden von \textsf{XCSoar} viele automatische Funktionen angeboten - nutze diese oder --falls nicht vorhanden-- 
frage z.B. das Entwicklerteam,  solche Funktionen zu entwickeln/ implementieren).   
\end{itemize}
