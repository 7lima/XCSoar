
\begin{center}
{\large GNU General Public License}



Deutsche Übersetzung der Version 2, Juni 1991


{\sc Copyright © 1989, 1991 Free Software Foundation, Inc.\\
51 Franklin St, Fifth Floor, Boston, MA 02110, USA}
\end{center}

\vspace{0.75em}
Es ist {\sl jedermann} gestattet, diese Lizenzurkunde zu vervielfältigen und unveränderte Kopien zu verbreiten; Änderungen sind jedoch nicht erlaubt.


Diese Übersetzung ist kein rechtskräftiger Ersatz für die englischsprachige Originalversion! {\small (welche im Anhang beigefügt ist)}

\vspace{1em}

{\bf Vorwort}


{\small

Die meisten Softwarelizenzen sind daraufhin entworfen worden, Ihnen die Freiheit zu \textit{nehmen}, die Software weiterzugeben und zu verändern. Im Gegensatz dazu soll Ihnen die GNU General Public License, die Allgemeine Öffentliche GNU-Lizenz, ebendiese Freiheit garantieren. Sie soll sicherstellen, daß die Software für alle Benutzer frei ist. Diese Lizenz gilt für den Großteil der von der Free Software Foundation herausgegebenen Software und für alle anderen Programme, deren Autoren ihr Werk dieser Lizenz unterstellt haben. Auch Sie können diese Möglichkeit der Lizenzierung für Ihre Programme anwenden. (Ein anderer Teil der Software der Free Software Foundation unterliegt stattdessen der GNU Lesser General Public License, der Kleineren Allgemeinen Öffentlichen GNU-Lizenz.)

Die Bezeichnung "freie" Software bezieht sich auf Freiheit, nicht auf den Preis. Unsere Lizenzen sollen Ihnen die Freiheit garantieren, Kopien freier Software zu verbreiten (und etwas für diesen Service zu berechnen, wenn Sie möchten), die Möglichkeit, die Software im Quelltext zu erhalten oder den Quelltext auf Wunsch zu bekommen. Die Lizenzen sollen garantieren, daß Sie die Software ändern oder Teile davon in neuen freien Programmen verwenden dürfen und daß Sie wissen, daß Sie dies alles tun dürfen.

Um Ihre Rechte zu schützen, müssen wir Einschränkungen machen, die es jedem verbieten, Ihnen diese Rechte zu verweigern oder Sie aufzufordern, auf diese Rechte zu verzichten. Aus diesen Einschränkungen folgen bestimmte Verantwortlichkeiten für Sie, wenn Sie Kopien der Software verbreiten oder sie verändern.

Beispielsweise müssen Sie den Empfängern alle Rechte gewähren, die Sie selbst haben, wenn Sie "kostenlos oder gegen Bezahlung" Kopien eines solchen Programms verbreiten. Sie müssen sicherstellen, daß auch die Empfänger den Quelltext erhalten bzw. erhalten können. Und Sie müssen ihnen diese Bedingungen zeigen, damit sie ihre Rechte kennen.

Wir schützen Ihre Rechte in zwei Schritten: 

(1) Wir stellen die Software unter ein Urheberrecht (Copyright), und 

(2) wir bieten Ihnen diese Lizenz an, die Ihnen das Recht gibt, die Software zu vervielfältigen, zu verbreiten und/oder zu verändern.

Um die Autoren und uns zu schützen, wollen wir darüberhinaus sicherstellen, daß jeder erfährt, daß für diese freie Software keinerlei Garantie besteht. Wenn die Software von jemand anderem modifiziert und weitergegeben wird, möchten wir, daß die Empfänger wissen, daß sie nicht das Original erhalten haben, damit irgendwelche von anderen verursachte Probleme nicht den Ruf des ursprünglichen Autors schädigen.

Schließlich und endlich ist jedes freie Programm permanent durch Software-Patente bedroht. Wir möchten die Gefahr ausschließen, daß Distributoren eines freien Programms individuell Patente lizensieren mit dem Ergebnis, daß das Programm proprietär würde. Um dies zu verhindern, haben wir klargestellt, daß jedes Patent entweder für freie Benutzung durch jedermann lizenziert werden muß oder überhaupt nicht lizenziert werden darf.


Es folgen die genauen Bedingungen für die Vervielfältigung, Verbreitung und Bearbeitung:



{\it Allgemeine Öffentliche GNU-Lizenz}

Bedingungen für die Vervielfältigung, Verbreitung und Bearbeitung
\begin{enumerate}
  \item Diese Lizenz gilt für jedes Programm und jedes andere Werk, in dem ein entsprechender Vermerk des Copyright-Inhabers darauf hinweist, daß das Werk unter den Bestimmungen dieser General Public License verbreitet werden darf. Im folgenden wird jedes derartige Programm oder Werk als "das Programm" bezeichnet; die Formulierung "auf dem Programm basierendes Werk" bezeichnet das Programm sowie jegliche Bearbeitung des Programms im urheberrechtlichen Sinne, also ein Werk, welches das Programm, auch auszugsweise, sei es unverändert oder verändert und/oder in eine andere Sprache übersetzt, enthält. (Im folgenden wird die Übersetzung ohne Einschränkung als "Bearbeitung" eingestuft.) Jeder Lizenznehmer wird im folgenden als "Sie" angesprochen.

Andere Handlungen als Vervielfältigung, Verbreitung und Bearbeitung werden von dieser Lizenz nicht berührt; sie fallen nicht in ihren Anwendungsbereich. Der Vorgang der Ausführung des Programms wird nicht eingeschränkt, und die Ausgaben des Programms unterliegen dieser Lizenz nur, wenn der Inhalt ein auf dem Programm basierendes Werk darstellt (unabhängig davon, daß die Ausgabe durch die Ausführung des Programmes erfolgte). Ob dies zutrifft, hängt von den Funktionen des Programms ab.

\item Sie dürfen auf beliebigen Medien unveränderte Kopien des Quelltextes des Programms, wie sie ihn erhalten haben, anfertigen und verbreiten. Voraussetzung hierfür ist, daß Sie mit jeder Kopie einen entsprechenden Copyright-Vermerk sowie einen Haftungsausschluß veröffentlichen, alle Vermerke, die sich auf diese Lizenz und das Fehlen einer Garantie beziehen, unverändert lassen und desweiteren allen anderen Empfängern des Programms zusammen mit dem Programm eine Kopie dieser Lizenz zukommen lassen.

Sie dürfen für den physikalischen Vorgang des Zugänglichmachens einer Kopie eine Gebühr verlangen. Wenn Sie es wünschen, dürfen Sie auch gegen Entgelt eine Garantie für das Programm anbieten.

\item Sie dürfen Ihre Kopie(n) des Programms oder eines Teils davon verändern, wodurch ein auf dem Programm basierendes Werk entsteht; Sie dürfen derartige Bearbeitungen unter den Bestimmungen von Paragraph 1 vervielfältigen und verbreiten, vorausgesetzt, daß zusätzlich alle im folgenden genannten Bedingungen erfüllt werden:

    Sie müssen die veränderten Dateien mit einem auffälligen Vermerk versehen, der auf die von Ihnen vorgenommene Modifizierung und das Datum jeder Änderung hinweist.

    Sie müssen dafür sorgen, daß jede von Ihnen verbreitete oder veröffentlichte Arbeit, die ganz oder teilweise von dem Programm oder Teilen davon abgeleitet ist, Dritten gegenüber als Ganzes unter den Bedingungen dieser Lizenz ohne Lizenzgebühren zur Verfügung gestellt wird.

    Wenn das veränderte Programm normalerweise bei der Ausführung interaktiv Kommandos einliest, müssen Sie dafür sorgen, daß es, wenn es auf dem üblichsten Wege für solche interaktive Nutzung gestartet wird, eine Meldung ausgibt oder ausdruckt, die einen geeigneten Copyright-Vermerk enthält sowie einen Hinweis, daß es keine Gewährleistung gibt (oder anderenfalls, daß Sie Garantie leisten), und daß die Benutzer das Programm unter diesen Bedingungen weiter verbreiten dürfen. Auch muß der Benutzer darauf hingewiesen werden, wie er eine Kopie dieser Lizenz ansehen kann. (Ausnahme: Wenn das Programm selbst interaktiv arbeitet, aber normalerweise keine derartige Meldung ausgibt, muß Ihr auf dem Programm basierendes Werk auch keine solche Meldung ausgeben).

Diese Anforderungen gelten für das bearbeitete Werk als Ganzes. Wenn identifizierbare Teile des Werkes nicht von dem Programm abgeleitet sind und vernünftigerweise als unabhängige und eigenständige Werke für sich selbst zu betrachten sind, dann gelten diese Lizenz und ihre Bedingungen nicht für die betroffenen Teile, wenn Sie diese als eigenständige Werke weitergeben. Wenn Sie jedoch dieselben Abschnitte als Teil eines Ganzen weitergeben, das ein auf dem Programm basierendes Werk darstellt, dann muß die Weitergabe des Ganzen nach den Bedingungen dieser Lizenz erfolgen, deren Bedingungen für weitere Lizenznehmer somit auf das gesamte Ganze ausgedehnt werden --- und somit auf jeden einzelnen Teil, unabhängig vom jeweiligen Autor.

Somit ist es nicht die Absicht dieses Abschnittes, Rechte für Werke in Anspruch zu nehmen oder Ihnen die Rechte für Werke streitig zu machen, die komplett von Ihnen geschrieben wurden; vielmehr ist es die Absicht, die Rechte zur Kontrolle der Verbreitung von Werken, die auf dem Programm basieren oder unter seiner auszugsweisen Verwendung zusammengestellt worden sind, auszuüben.

Ferner bringt auch das einfache Zusammenlegen eines anderen Werkes, das nicht auf dem Programm basiert, mit dem Programm oder einem auf dem Programm basierenden Werk auf ein- und demselben Speicher- oder Vertriebsmedium dieses andere Werk nicht in den Anwendungsbereich dieser Lizenz.

\item Sie dürfen das Programm (oder ein darauf basierendes Werk gemäß Paragraph 2) als Objectcode oder in ausführbarer Form unter den Bedingungen der Paragraphen 1 und 2 kopieren und weitergeben vorausgesetzt, daß Sie außerdem eine der folgenden Leistungen erbringen:
\begin{itemize}
    \item[a] Liefern Sie das Programm zusammen mit dem vollständigen zugehörigen maschinenlesbaren Quelltext auf einem für den Datenaustausch üblichen Medium aus, wobei die Verteilung unter den Bedingungen der Paragraphen 1 und 2 erfolgen muß. Oder:
    \item[b]Liefern Sie das Programm zusammen mit einem mindestens drei Jahre lang gültigen schriftlichen Angebot aus, jedem Dritten eine vollständige maschinenlesbare Kopie des Quelltextes zur Verfügung zu stellen "zu nicht höheren Kosten als denen, die durch das physikalische Zugänglichmachen des Quelltextes anfallen", wobei der Quelltext unter den Bedingungen der Paragraphen 1 und 2 auf einem für den Datenaustausch üblichen Medium weitergegeben wird. Oder:
    \item[c]Liefern Sie das Programm zusammen mit dem schriftlichen Angebot der Zurverfügungstellung des Quelltextes aus, das Sie selbst erhalten haben. (Diese Alternative ist nur für nicht-kommerzielle Verbreitung zulässig und nur, wenn Sie das Programm als Objectcode oder in ausführbarer Form mit einem entsprechenden Angebot gemäß Absatz b erhalten haben.)
\end{itemize}
Unter dem Quelltext eines Werkes wird diejenige Form des Werkes verstanden, die für Bearbeitungen vorzugsweise verwendet wird. Für ein ausführbares Programm bedeutet "der komplette Quelltext": Der Quelltext aller im Programm enthaltenen Module einschließlich aller zugehörigen Modulschnittstellen-Definitionsdateien sowie der zur Compilation und Installation verwendeten Skripte. Als besondere Ausnahme jedoch braucht der verteilte Quelltext nichts von dem zu enthalten, was üblicherweise (entweder als Quelltext oder in binärer Form) zusammen mit den Hauptkomponenten des Betriebssystems (Kernel, Compiler usw.) geliefert wird, unter dem das Programm läuft --- es sei denn, diese Komponente selbst gehört zum ausführbaren Programm.

Wenn die Verbreitung eines ausführbaren Programms oder von Objectcode dadurch erfolgt, daß der Kopierzugriff auf eine dafür vorgesehene Stelle gewährt wird, so gilt die Gewährung eines gleichwertigen Kopierzugriffs auf den Quelltext von derselben Stelle als Verbreitung des Quelltextes, auch wenn Dritte nicht dazu gezwungen sind, den Quelltext zusammen mit dem Objectcode zu kopieren.

\item Sie dürfen das Programm nicht vervielfältigen, verändern, weiter lizenzieren oder verbreiten, sofern es nicht durch diese Lizenz ausdrücklich gestattet ist. Jeder anderweitige Versuch der Vervielfältigung, Modifizierung, Weiterlizenzierung und Verbreitung ist nichtig und beendet automatisch Ihre Rechte unter dieser Lizenz. Jedoch werden die Lizenzen Dritter, die von Ihnen Kopien oder Rechte unter dieser Lizenz erhalten haben, nicht beendet, solange diese die Lizenz voll anerkennen und befolgen.

\item  Sie sind nicht verpflichtet, diese Lizenz anzunehmen, da Sie sie nicht unterzeichnet haben. Jedoch gibt Ihnen nichts anderes die Erlaubnis, das Programm oder von ihm abgeleitete Werke zu verändern oder zu verbreiten. Diese Handlungen sind gesetzlich verboten, wenn Sie diese Lizenz nicht anerkennen. Indem Sie das Programm (oder ein darauf basierendes Werk) verändern oder verbreiten, erklären Sie Ihr Einverständnis mit dieser Lizenz und mit allen ihren Bedingungen bezüglich der Vervielfältigung, Verbreitung und Veränderung des Programms oder eines darauf basierenden Werks.

\item  Jedesmal, wenn Sie das Programm (oder ein auf dem Programm basierendes Werk) weitergeben, erhält der Empfänger automatisch vom ursprünglichen Lizenzgeber die Lizenz, das Programm entsprechend den hier festgelegten Bestimmungen zu vervielfältigen, zu verbreiten und zu verändern. Sie dürfen keine weiteren Einschränkungen der Durchsetzung der hierin zugestandenen Rechte des Empfängers vornehmen. Sie sind nicht dafür verantwortlich, die Einhaltung dieser Lizenz durch Dritte durchzusetzen.

\item Sollten Ihnen infolge eines Gerichtsurteils, des Vorwurfs einer Patentverletzung oder aus einem anderen Grunde (nicht auf Patentfragen begrenzt) Bedingungen (durch Gerichtsbeschluß, Vergleich oder anderweitig) auferlegt werden, die den Bedingungen dieser Lizenz widersprechen, so befreien Sie diese Umstände nicht von den Bestimmungen dieser Lizenz. Wenn es Ihnen nicht möglich ist, das Programm unter gleichzeitiger Beachtung der Bedingungen in dieser Lizenz und Ihrer anderweitigen Verpflichtungen zu verbreiten, dann dürfen Sie als Folge das Programm überhaupt nicht verbreiten. Wenn zum Beispiel ein Patent nicht die gebührenfreie Weiterverbreitung des Programms durch diejenigen erlaubt, die das Programm direkt oder indirekt von Ihnen erhalten haben, dann besteht der einzige Weg, sowohl das Patentrecht als auch diese Lizenz zu befolgen, darin, ganz auf die Verbreitung des Programms zu verzichten.

Sollte sich ein Teil dieses Paragraphen als ungültig oder unter bestimmten Umständen nicht durchsetzbar erweisen, so soll dieser Paragraph seinem Sinne nach angewandt werden; im übrigen soll dieser Paragraph als Ganzes gelten.

Zweck dieses Paragraphen ist nicht, Sie dazu zu bringen, irgendwelche Patente oder andere Eigentumsansprüche zu verletzen oder die Gültigkeit solcher Ansprüche zu bestreiten; dieser Paragraph hat einzig den Zweck, die Integrität des Verbreitungssystems der freien Software zu schützen, das durch die Praxis öffentlicher Lizenzen verwirklicht wird. Viele Leute haben großzügige Beiträge zu dem großen Angebot der mit diesem System verbreiteten Software im Vertrauen auf die konsistente Anwendung dieses Systems geleistet; es liegt am Autor/Geber, zu entscheiden, ob er die Software mittels irgendeines anderen Systems verbreiten will; ein Lizenznehmer hat auf diese Entscheidung keinen Einfluß.

Dieser Paragraph ist dazu gedacht, deutlich klarzustellen, was als Konsequenz aus dem Rest dieser Lizenz betrachtet wird.

\item Wenn die Verbreitung und/oder die Benutzung des Programms in bestimmten Staaten entweder durch Patente oder durch urheberrechtlich geschützte Schnittstellen eingeschränkt ist, kann der Urheberrechtsinhaber, der das Programm unter diese Lizenz gestellt hat, eine explizite geographische Begrenzung der Verbreitung angeben, in der diese Staaten ausgeschlossen werden, so daß die Verbreitung nur innerhalb und zwischen den Staaten erlaubt ist, die nicht ausgeschlossen sind. In einem solchen Fall beinhaltet diese Lizenz die Beschränkung, als wäre sie in diesem Text niedergeschrieben.

\item Die Free Software Foundation kann von Zeit zu Zeit überarbeitete und/oder neue Versionen der General Public License veröffentlichen. Solche neuen Versionen werden vom Grundprinzip her der gegenwärtigen entsprechen, können aber im Detail abweichen, um neuen Problemen und Anforderungen gerecht zu werden.

Jede Version dieser Lizenz hat eine eindeutige Versionsnummer. Wenn in einem Programm angegeben wird, daß es dieser Lizenz in einer bestimmten Versionsnummer oder "jeder späteren Version" ("any later version") unterliegt, so haben Sie die Wahl, entweder den Bestimmungen der genannten Version zu folgen oder denen jeder beliebigen späteren Version, die von der Free Software Foundation veröffentlicht wurde. Wenn das Programm keine Versionsnummer angibt, können Sie eine beliebige Version wählen, die je von der Free Software Foundation veröffentlicht wurde.

\item Wenn Sie den Wunsch haben, Teile des Programms in anderen freien Programmen zu verwenden, deren Bedingungen für die Verbreitung anders sind, schreiben Sie an den Autor, um ihn um die Erlaubnis zu bitten. Für Software, die unter dem Copyright der Free Software Foundation steht, schreiben Sie an die Free Software Foundation; wir machen zu diesem Zweck gelegentlich Ausnahmen. Unsere Entscheidung wird von den beiden Zielen geleitet werden, zum einen den freien Status aller von unserer freien Software abgeleiteten Werke zu erhalten und zum anderen das gemeinschaftliche Nutzen und Wiederverwenden von Software im allgemeinen zu fördern.
Keine Gewährleistung
\end{enumerate}

{\sc
Da das Programm ohne jegliche Kosten lizenziert wird, besteht keinerlei Gewährleistung für das Programm, soweit dies gesetzlich zulässig ist. Sofern nicht anderweitig schriftlich bestätigt, stellen die Copyright-Inhaber und/oder Dritte das Programm so zur Verfügung, "wie es ist", ohne irgendeine Gewährleistung, weder ausdrücklich noch implizit, einschließlich --- aber nicht begrenzt auf --- Marktreife oder Verwendbarkeit für einen bestimmten Zweck. Das volle Risiko bezüglich Qualität und Leistungsfähigkeit des Programms liegt bei Ihnen. Sollte sich das Programm als fehlerhaft herausstellen, liegen die Kosten für notwendigen Service, Reparatur oder Korrektur bei Ihnen.

In keinem Fall, außer wenn durch geltendes Recht gefordert oder schriftlich zugesichert, ist irgendein Copyright-Inhaber oder irgendein Dritter, der das Programm wie oben erlaubt modifiziert oder verbreitet hat, Ihnen gegenüber für irgendwelche Schäden haftbar, einschließlich jeglicher allgemeiner oder spezieller Schäden, Schäden durch Seiteneffekte (Nebenwirkungen) oder Folgeschäden, die aus der Benutzung des Programms oder der Unbenutzbarkeit des Programms folgen (einschließlich --- aber nicht beschränkt auf Datenverluste --- fehlerhafte Verarbeitung von Daten, Verluste, die von Ihnen oder anderen getragen werden müssen, oder dem Unvermögen des Programms, mit irgendeinem anderen Programm zusammenzuarbeiten), selbst wenn ein Copyright-Inhaber oder Dritter über die Möglichkeit solcher Schäden unterrichtet worden war. 
}
}


\vspace{1cm}
Auf den folgenden Seiten im Anschluß das englische Original der General Public License: 
\newpage 

{\small
\begin{center}
		    {\large GNU GENERAL PUBLIC LICENSE}

		       Version 2, June 1991


 {\sc Copyright (C) 1989, 1991 Free Software Foundation, Inc.\\
 59 Temple Place, Suite 330, Boston, MA  02111-1307  USA}
\end{center}

\vspace{0.75em}

 Everyone is permitted to copy and distribute verbatim copies
 of this license document, but changing it is not allowed.


\vspace{0.75cm}

{\bf Preamble}

\vspace{0.5cm}

  The licenses for most software are designed to take away your
freedom to share and change it.  By contrast, the GNU General Public
License is intended to guarantee your freedom to share and change free
software--to make sure the software is free for all its users.  This
General Public License applies to most of the Free Software
Foundation's software and to any other program whose authors commit to
using it.  (Some other Free Software Foundation software is covered by
the GNU Library General Public License instead.)  You can apply it to
your programs, too.

  When we speak of free software, we are referring to freedom, not
price.  Our General Public Licenses are designed to make sure that you
have the freedom to distribute copies of free software (and charge for
this service if you wish), that you receive source code or can get it
if you want it, that you can change the software or use pieces of it
in new free programs; and that you know you can do these things.

  To protect your rights, we need to make restrictions that forbid
anyone to deny you these rights or to ask you to surrender the rights.
These restrictions translate to certain responsibilities for you if you
distribute copies of the software, or if you modify it.

  For example, if you distribute copies of such a program, whether
gratis or for a fee, you must give the recipients all the rights that
you have.  You must make sure that they, too, receive or can get the
source code.  And you must show them these terms so they know their
rights.

  We protect your rights with two steps: (1) copyright the software, and
(2) offer you this license which gives you legal permission to copy,
distribute and/or modify the software.

  Also, for each author's protection and ours, we want to make certain
that everyone understands that there is no warranty for this free
software.  If the software is modified by someone else and passed on, we
want its recipients to know that what they have is not the original, so
that any problems introduced by others will not reflect on the original
authors' reputations.

  Finally, any free program is threatened constantly by software
patents.  We wish to avoid the danger that redistributors of a free
program will individually obtain patent licenses, in effect making the
program proprietary.  To prevent this, we have made it clear that any
patent must be licensed for everyone's free use or not licensed at all.

  The precise terms and conditions for copying, distribution and
modification follow.

\begin{center}
		    GNU GENERAL PUBLIC LICENSE

   TERMS AND CONDITIONS FOR COPYING, DISTRIBUTION AND MODIFICATION
\end{center}

\begin{enumerate}
\item This License applies to any program or other work which contains
a notice placed by the copyright holder saying it may be distributed
under the terms of this General Public License.  The "Program", below,
refers to any such program or work, and a "work based on the Program"
means either the Program or any derivative work under copyright law:
that is to say, a work containing the Program or a portion of it,
either verbatim or with modifications and/or translated into another
language.  (Hereinafter, translation is included without limitation in
the term "modification".)  Each licensee is addressed as "you".

Activities other than copying, distribution and modification are not
covered by this License; they are outside its scope.  The act of
running the Program is not restricted, and the output from the Program
is covered only if its contents constitute a work based on the
Program (independent of having been made by running the Program).
Whether that is true depends on what the Program does.

\item You may copy and distribute verbatim copies of the Program's
source code as you receive it, in any medium, provided that you
conspicuously and appropriately publish on each copy an appropriate
copyright notice and disclaimer of warranty; keep intact all the
notices that refer to this License and to the absence of any warranty;
and give any other recipients of the Program a copy of this License
along with the Program.

You may charge a fee for the physical act of transferring a copy, and
you may at your option offer warranty protection in exchange for a fee.

\item You may modify your copy or copies of the Program or any portion
of it, thus forming a work based on the Program, and copy and
distribute such modifications or work under the terms of Section 1
above, provided that you also meet all of these conditions:

\begin{enumerate}
    \item You must cause the modified files to carry prominent notices
    stating that you changed the files and the date of any change.

    \item You must cause any work that you distribute or publish, that in
    whole or in part contains or is derived from the Program or any
    part thereof, to be licensed as a whole at no charge to all third
    parties under the terms of this License.

    \item If the modified program normally reads commands interactively
    when run, you must cause it, when started running for such
    interactive use in the most ordinary way, to print or display an
    announcement including an appropriate copyright notice and a
    notice that there is no warranty (or else, saying that you provide
    a warranty) and that users may redistribute the program under
    these conditions, and telling the user how to view a copy of this
    License.  (Exception: if the Program itself is interactive but
    does not normally print such an announcement, your work based on
    the Program is not required to print an announcement.)
\end{enumerate}

These requirements apply to the modified work as a whole.  If
identifiable sections of that work are not derived from the Program,
and can be reasonably considered independent and separate works in
themselves, then this License, and its terms, do not apply to those
sections when you distribute them as separate works.  But when you
distribute the same sections as part of a whole which is a work based
on the Program, the distribution of the whole must be on the terms of
this License, whose permissions for other licensees extend to the
entire whole, and thus to each and every part regardless of who wrote it.

Thus, it is not the intent of this section to claim rights or contest
your rights to work written entirely by you; rather, the intent is to
exercise the right to control the distribution of derivative or
collective works based on the Program.

In addition, mere aggregation of another work not based on the Program
with the Program (or with a work based on the Program) on a volume of
a storage or distribution medium does not bring the other work under
the scope of this License.

\item You may copy and distribute the Program (or a work based on it,
under Section 2) in object code or executable form under the terms of
Sections 1 and 2 above provided that you also do one of the following:

\begin{enumerate}
    \item Accompany it with the complete corresponding machine-readable
    source code, which must be distributed under the terms of Sections
    1 and 2 above on a medium customarily used for software interchange; or,

    \item Accompany it with a written offer, valid for at least three
    years, to give any third party, for a charge no more than your
    cost of physically performing source distribution, a complete
    machine-readable copy of the corresponding source code, to be
    distributed under the terms of Sections 1 and 2 above on a medium
    customarily used for software interchange; or,

    \item Accompany it with the information you received as to the offer
    to distribute corresponding source code.  (This alternative is
    allowed only for noncommercial distribution and only if you
    received the program in object code or executable form with such
    an offer, in accord with Subsection b above.)
\end{enumerate}

The source code for a work means the preferred form of the work for
making modifications to it.  For an executable work, complete source
code means all the source code for all modules it contains, plus any
associated interface definition files, plus the scripts used to
control compilation and installation of the executable.  However, as a
special exception, the source code distributed need not include
anything that is normally distributed (in either source or binary
form) with the major components (compiler, kernel, and so on) of the
operating system on which the executable runs, unless that component
itself accompanies the executable.

If distribution of executable or object code is made by offering
access to copy from a designated place, then offering equivalent
access to copy the source code from the same place counts as
distribution of the source code, even though third parties are not
compelled to copy the source along with the object code.

\item You may not copy, modify, sublicense, or distribute the Program
except as expressly provided under this License.  Any attempt
otherwise to copy, modify, sublicense or distribute the Program is
void, and will automatically terminate your rights under this License.
However, parties who have received copies, or rights, from you under
this License will not have their licenses terminated so long as such
parties remain in full compliance.

\item You are not required to accept this License, since you have not
signed it.  However, nothing else grants you permission to modify or
distribute the Program or its derivative works.  These actions are
prohibited by law if you do not accept this License.  Therefore, by
modifying or distributing the Program (or any work based on the
Program), you indicate your acceptance of this License to do so, and
all its terms and conditions for copying, distributing or modifying
the Program or works based on it.

\item Each time you redistribute the Program (or any work based on the
Program), the recipient automatically receives a license from the
original licensor to copy, distribute or modify the Program subject to
these terms and conditions.  You may not impose any further
restrictions on the recipients' exercise of the rights granted herein.
You are not responsible for enforcing compliance by third parties to
this License.

\item If, as a consequence of a court judgment or allegation of patent
infringement or for any other reason (not limited to patent issues),
conditions are imposed on you (whether by court order, agreement or
otherwise) that contradict the conditions of this License, they do not
excuse you from the conditions of this License.  If you cannot
distribute so as to satisfy simultaneously your obligations under this
License and any other pertinent obligations, then as a consequence you
may not distribute the Program at all.  For example, if a patent
license would not permit royalty-free redistribution of the Program by
all those who receive copies directly or indirectly through you, then
the only way you could satisfy both it and this License would be to
refrain entirely from distribution of the Program.

If any portion of this section is held invalid or unenforceable under
any particular circumstance, the balance of the section is intended to
apply and the section as a whole is intended to apply in other
circumstances.

It is not the purpose of this section to induce you to infringe any
patents or other property right claims or to contest validity of any
such claims; this section has the sole purpose of protecting the
integrity of the free software distribution system, which is
implemented by public license practices.  Many people have made
generous contributions to the wide range of software distributed
through that system in reliance on consistent application of that
system; it is up to the author/donor to decide if he or she is willing
to distribute software through any other system and a licensee cannot
impose that choice.

This section is intended to make thoroughly clear what is believed to
be a consequence of the rest of this License.

\item If the distribution and/or use of the Program is restricted in
certain countries either by patents or by copyrighted interfaces, the
original copyright holder who places the Program under this License
may add an explicit geographical distribution limitation excluding
those countries, so that distribution is permitted only in or among
countries not thus excluded.  In such case, this License incorporates
the limitation as if written in the body of this License.

\item The Free Software Foundation may publish revised and/or new versions
of the General Public License from time to time.  Such new versions will
be similar in spirit to the present version, but may differ in detail to
address new problems or concerns.

Each version is given a distinguishing version number.  If the Program
specifies a version number of this License which applies to it and "any
later version", you have the option of following the terms and conditions
either of that version or of any later version published by the Free
Software Foundation.  If the Program does not specify a version number of
this License, you may choose any version ever published by the Free Software
Foundation.

\item If you wish to incorporate parts of the Program into other free
programs whose distribution conditions are different, write to the author
to ask for permission.  For software which is copyrighted by the Free
Software Foundation, write to the Free Software Foundation; we sometimes
make exceptions for this.  Our decision will be guided by the two goals
of preserving the free status of all derivatives of our free software and
of promoting the sharing and reuse of software generally.

\end{enumerate}

\subsection*{No warranty}
{\sc 
 BECAUSE THE PROGRAM IS LICENSED FREE OF CHARGE, THERE IS NO WARRANTY
 FOR THE PROGRAM, TO THE EXTENT PERMITTED BY APPLICABLE LAW.  EXCEPT
 WHEN OTHERWISE STATED IN WRITING THE COPYRIGHT HOLDERS AND/OR OTHER
 PARTIES PROVIDE THE PROGRAM "AS IS" WITHOUT WARRANTY OF ANY KIND,
 EITHER EXPRESSED OR IMPLIED, INCLUDING, BUT NOT LIMITED TO, THE
 IMPLIED WARRANTIES OF MERCHANTABILITY AND FITNESS FOR A PARTICULAR
 PURPOSE.  THE ENTIRE RISK AS TO THE QUALITY AND PERFORMANCE OF THE
 PROGRAM IS WITH YOU.  SHOULD THE PROGRAM PROVE DEFECTIVE, YOU ASSUME
 THE COST OF ALL NECESSARY SERVICING, REPAIR OR CORRECTION.

 IN NO EVENT UNLESS REQUIRED BY APPLICABLE LAW OR AGREED TO IN WRITING
 WILL ANY COPYRIGHT HOLDER, OR ANY OTHER PARTY WHO MAY MODIFY AND/OR
 REDISTRIBUTE THE PROGRAM AS PERMITTED ABOVE, BE LIABLE TO YOU FOR
 DAMAGES, INCLUDING ANY GENERAL, SPECIAL, INCIDENTAL OR CONSEQUENTIAL
 DAMAGES ARISING OUT OF THE USE OR INABILITY TO USE THE PROGRAM
 (INCLUDING BUT NOT LIMITED TO LOSS OF DATA OR DATA BEING RENDERED
 INACCURATE OR LOSSES SUSTAINED BY YOU OR THIRD PARTIES OR A FAILURE
 OF THE PROGRAM TO OPERATE WITH ANY OTHER PROGRAMS), EVEN IF SUCH
 HOLDER OR OTHER PARTY HAS BEEN ADVISED OF THE POSSIBILITY OF SUCH
 DAMAGES.
}
}





\endinput
