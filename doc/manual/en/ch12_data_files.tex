\chapter{Data Files}\label{cha:data-files}

Data files used by XCSoar fall into two categories:
\begin{description}
\item[Flight data files]  These files contain data relating to
the aircraft type, airspace and maps, waypoints etc.  These are the
files that are most likely to be modified or set by normal users.
\item[Program data files]  These files contain data relating to
the `look and feel' of the program, including language translations, 
button assignments, input events, dialog layouts.
\end{description}
This chapter focuses on flight data files; see the {\em XCSoar
Advanced Configuration Guide} for details on program data files.

\section{File management}

File names must correspond to the name extensions specified below.  It
is good practice to make sure that the file names are recognisable so
that when making configuration changes there is less risk of confusion
between different files and different file types.

Although it is a generally good idea to have data files located in
nonvolatile memory, the use of SD cards and other removable media in
PDAs can cause performance issues; for smaller files, and files that
are only accessed at start-up (waypoints, airspace, glide polars,
configuration files), this is acceptable.  However, terrain and
topology files are accessed continuously while XCSoar is running, so
these should be located in faster storage memory.

Many PDAs provide a 'file store' which is nonvolatile; the same
arguments above apply regarding their use and performance.

All data files should be copied into the directory:
\begin{verbatim}
My Documents/XCSoarData
\end{verbatim}

For PDA users, data can also be stored on the operating system file
store, on Compact Flash cards or SD cards under the directory
\verb|XCSoarData|.

For example:
\begin{verbatim}
SD Card/XCSoarData
IPAQ File Store/XCSoarData
\end{verbatim}

\todonum[inline]{Explain where files are generated to and search paths.}

Note that due to limitations in the Pocket PC and Windows Mobile
operating system, additional subdirectories under ``My
Documents/XCSoarData'' are not allowed.

\section{Composed Mapfile}

\todonum[inline]{What is the new xcm file format capable, and what does it
replace}

\section{Terrain}

The terrain file (extension \verb|.dat|) is a raster digital elevation
model represented as an array of elevations in meters on a
latitude/longitude grid.  The format used is unique to XCSoar as it
contains a header containing the grid geometry followed by the raster
array.
\todonum{xcm impact?}

Terrain files for various regions can be obtained from the XCSoar website.
Additional terrain files can be produced upon request.

\section{Topology}

The topology file (extension \verb|.tpl|) is a text file containing a
series of entries each of which define a layer of topology.  Typical
layers include roads, railway lines, large built-up areas (cities),
miscellaneous populated areas (towns and villages), lakes and rivers.

The topology file defines which features are to be displayed, their
colour, maximum zoom visibility, icons, and labelling.  This file can
be customised, for example to add or remove specific layers.  Details
on the file format will be provided in a separate document.  The
topology data itself uses ERSI Shape files which are generated from the
freely available VMAP0 database.
\todonum{xcm impact?}

Topology files for various regions can be obtained from the XCSoar
website.  Additional topology files can be produced upon request.

\section{Waypoints}

XCSoar uses waypoint files written in the format designed by Cambridge
Aero Instruments for their C302 instrument.  The file extension should
be \verb|.dat|.
\todonum{add new waypoint formats}
Files are available from the Soaring Turn-points section of the
Soaring Server\footnote{Mirrors to this website exist, google search
for ``worldwide soaring turnpoint exchange'' if the main server is
inaccessible.}:

\verb|http://acro.harvard.edu/SOARING/JL/TP|

Several commercial and freely distributable programs exist for
converting between different waypoint formats.

If the elevation of any waypoints is set to zero in the waypoint file,
then XCSoar estimates the waypoint elevation from the terrain database
if available.

\section{Airspace}

XCSoar supports airspace files (extension \verb|.txt|) using a sub set
of the widely distributed OpenAir format. Files are available from the
Special Use Airspace section of the Soaring Server:

\verb|http://acro.harvard.edu/SOARING/JL/SUA|

\todonum[inline]{support for TNP airspace files}
\todonum[inline]{interference, preference to airspace data in xcm file?}

The following are the list of supported airspace types: Class A, Class
B, Class C, Class D, Class E, Class F, Prohibited areas, Danger areas,
Restricted Areas, CTR, No Gliders, Wave, Other.  All other airspace
types will be drawn as type ``Other''.

\section{Map}\label{sec:map}

The map file (extension \verb|.xcm|) contains terrain, topology and
optionally waypoints and airspace information.  The use of map files
reduces the number of files the user needs to manage and to specify in
the configuration settings.  For backward compatibility, though, the
previous methods of using individual terrain, topology, and waypoint
files has been retained.

Map files can be generated from the online terrain/map file generator
available from the xcsoar website.  This allows users to generate
their own map files for their region, incorporating their own waypoint
files or by specifying the bounds of the region of interest.

Map files are superior to individual terrain/topology files because
they incorporate compression of the data, thereby allowing much larger
areas and higher resolution terrain to be used.

\section{Airfield details}

The airfield details file (extension \verb|.txt|) is a simple text
format file containing entries for each airfield, marked in square
brackets in uppercase, followed by the text to be displayed on the
Waypoint Details Dialog for that particular waypoint.  The text should
have a narrow margin because the waypoint details dialog cannot
currently handle word wrapping.

The names of airfields used in the file must correspond exactly to the
names in the waypoints file, with the exception of being converted to
uppercase.

The XCSoar website provides airfield details files for several
countries and includes tools to convert from various Enroute
Supplement sources to this file format.

Users are free to edit these files to add their own notes for
airfields that may not otherwise be included in the Enroute Supplement
sources.

An example (extract from the Australian airfields file):
\begin{verbatim}
[BENALLA]
RUNWAYS:
  08 (RL1,7) 17 (RL53) 26
  (R) 35 (R)

COMMUNICATIONS:
  CTAF - 122.5 REMARKS: Nstd
  10 NM rad to 5000'

REMARKS:
  CAUTION - Animal haz. Rwy
  08L-26R and 17L-35R for
  glider ops and tailskidacft
  only, SR-SS. TFC PAT - Rgt
  circuits Rwy 08R-26L. NS
  ABTMT - Rwy 17R-35L fly wide

ICAO: YBLA

[GROOTE EYLANDT]
Blah blah blah blah
...
\end{verbatim}

\section{Glide polar}

The WinPilot format is used for glide polar files (extension \verb|.plr|).

The WinPilot and XCSoar websites provide several glide polar files.
Files for other gliders may be created upon request to the XCSoar
team.

The format of the file is simple.  Lines beginning with \verb|*| are
ignored and so may be used to document how the polar was calculated or
if there are restrictions on its use.  Other than comments, the file
must contain a single row of numbers separated with commas:
\begin{itemize}
\item Mass dry gross weight in kg: this is the weight of the glider plus
  a 'standard' pilot without ballast.
\item Max water ballast in liters (kg)
\item Speed in km/h for first measurement point, (usually minimum sink speed)
\item Sink rate in m/s for first measurement point
\item Speed in km/h for second measurement point, (usually best glide speed)
\item Sink rate in m/s for second measurement point
\item Speed in km/h for third measurement point, (usually max manoeuvring speed)
\item Sink rate in m/s for third measurement point
\end{itemize}

An example, for the LS-3 glider, is given below:
\begin{verbatim}
*LS-3	WinPilot POLAR file: MassDryGross[kg], 
*  MaxWaterBallast[liters], Speed1[km/h], Sink1[m/s], 
*  Speed2, Sink2, Speed3, Sink3  	
373,	121,	74.1,	-0.65,	102.0,	-0.67,	167.0,	-1.85
\end{verbatim}

\tip Don't be too optimistic when entering your polar data. It is all too
easy to set your LD too high and you will rapidly see yourself
undershooting on final glide.

The polars built in to XCSoar are documented in the table below.

\begin{maxipage}
\begin{small}
\begin{longtable}{l l l l l l l l l}
\toprule
Name & Empty & Ballast & V1 & W1 & V2 & W2 & V3 & W3 \\
     & mass       & mass         &  &  &  &  &  &  \\
     & (kg)       & (kg)         & (kph) & (m/s) & (kph) & (m/s) & (kph) & (m/s) \\
\midrule
1-26E & 315 & 0 & 82.3 & -1.04 & 117.73 & -1.88 & 156.86 & -3.8 \\
1-34 & 354 & 0 & 89.82 & -0.8 & 143.71 & -2.1 & 179.64 & -3.8 \\
1-35A & 381 & 179 & 98.68 & -0.74 & 151.82 & -1.8 & 202.87 & -3.9 \\
1-36 Sprite & 322 & 0 & 75.98 & -0.68 & 132.96 & -2 & 170.95 & -4.1 \\
604 & 570 & 100 & 112.97 & 0.72 & 150.64 & -1.42 & 207.13 & -4.1 \\
ASH-25M 2 & 750 & 121 & 130.01 & -0.78 & 169.96 & -1.4 & 219.94 & -2.6 \\
ASH-25M 1 & 660 & 121 & 121.3 & -0.73 & 159.35 & -1.31 & 206.22 & -2.4 \\
ASH-25 (25m, PAS) & 693 & 120 & 105.67 & -0.56 & 163.25 & -1.34 & 211.26 & -2.5 \\
ASH-25 (25m, PIL) & 602 & 120 & 98.5 & -0.52 & 152.18 & -1.25 & 196.93 & -2.3 \\
AstirCS & 330 & 90 & 75.0 & -0.7 & 93.0 & -0.74 & 185.00 & -3.1 \\
ASW-12 & 948 & 189 & 95 & -0.57 & 148 & -1.48 & 183.09 & -2.6 \\
ASW-15 & 349 & 91 & 97.56 & -0.77 & 156.12 & -1.9 & 195.15 & -3.4 \\
ASW-17 & 522 & 151 & 114.5 & -0.7 & 169.05 & -1.68 & 206.5 & -2.9 \\
ASW-19 & 363 & 125 & 97.47 & -0.74 & 155.96 & -1.64 & 194.96 & -3.1 \\
ASW-20 & 377 & 159 & 116.2 & -0.77 & 174.3 & -1.89 & 213.04 & -3.3 \\
ASW-24 & 350 & 159 & 108.82 & -0.73 & 142.25 & -1.21 & 167.41 & -1.8 \\
ASW-27 Wnglts & 357 & 165 & 108.8 & -0.64 & 156.4 & -1.18 & 211.13 & -2.5 \\
Std Cirrus & 337 & 80 & 93.23 & -0.74 & 149.17 & -1.71 & 205.1 & -4.2 \\
Cobra & 350 & 30 & 70.8 & -0.60 & 94.5 & -0.69 & 148.1 & -1.83 \\
DG-400 (15m) & 440 & 90 & 115 & -0.76 & 160.53 & -1.22 & 210.22 & -2.3 \\
DG-400 (17m) & 444 & 90 & 118.28 & -0.68 & 163.77 & -1.15 & 198.35 & -1.8 \\
DG-500M PAS & 750 & 100 & 121.6 & -0.75 & 162.12 & -1.37 & 202.66 & -2.5 \\
DG-500M PIL & 659 & 100 & 115.4 & -0.71 & 152.01 & -1.28 & 190.02 & -2.3 \\
DG-500 PAS & 660 & 160 & 115.5 & -0.72 & 152.16 & -1.28 & 190.22 & -2.3 \\
DG-500 PIL & 570 & 160 & 107.5 & -0.66 & 141.33 & -1.19 & 176.66 & -2.1 \\
DG-800 15m & 468 & 120 & 133.9 & -0.88 & 178.87 & -1.53 & 223.59 & -2.5 \\
DG-800 18m Wnglts & 472 & 120 & 106 & -0.62 & 171.75 & -1.47 & 214.83 & -2.4 \\
Discus A & 350 & 182 & 103.77 & -0.72 & 155.65 & -1.55 & 190.24 & -3.1 \\
Duo Discus (PAS) & 628 & 201 & 106.5 & -0.79 & 168.11 & -1.54 & 201.31 & -2.9 \\
Duo Discus (PIL) & 537 & 201 & 94.06 & -0.72 & 155.49 & -1.43 & 188.21 & -2.7 \\
Genesis II & 374 & 151 & 94 & -0.61 & 141.05 & -1.18 & 172.4 & -2.0 \\
Grob G-103 Twin II (PAS) & 580 & 0 & 99 & -0.8 & 175.01 & -1.95 & 225.02 & -3.8 \\
Grob G-103 Twin II (PIL) & 494 & 0 & 90.75 & -0.74 & 161.42 & -1.8 & 207.54 & -3.5 \\
H-201 Std Libelle & 304 & 50 & 97 & -0.79 & 152.43 & -1.91 & 190.54 & -3.3 \\
H-301 Libelle & 300 & 50 & 94 & -0.68 & 147.71 & -2.03 & 184.64 & -4.1 \\
IS-29D2 Lark & 360 & 0 & 100 & -0.82 & 135.67 & -1.55 & 184.12 & -3.3 \\
Jantar 2 (SZD-42A) & 482 & 191 & 109.5 & -0.66 & 157.14 & -1.47 & 196.42 & -2.7 \\
Janus B (18.2m PAS) & 603 & 170 & 115.5 & -0.76 & 171.79 & -1.98 & 209.96 & -4.0 \\
Janus B (18.2m PIL) & 508 & 170 & 105.7 & -0.7 & 157.65 & -1.82 & 192.68 & -3.6 \\
Ka-6CR & 310 & 0 & 87.35 & -0.81 & 141.92 & -2.03 & 174.68 & -3.5 \\
L-33 SOLO & 330 & 0 & 87.2 & -0.8 & 135.64 & -1.73 & 174.4 & -3.4 \\
LS-1C & 350 & 91 & 115.87 & -1.02 & 154.49 & -1.84 & 193.12 & -3.3 \\
\bottomrule
\end{longtable}
\end{small}
\end{maxipage}

\begin{maxipage}
\begin{small}
\begin{longtable}{l l l l l l l l l}
\toprule
Name & Empty & Ballast & V1 & W1 & V2 & W2 & V3 & W3 \\
     & mass       & mass         &  &  &  &  &  &  \\
     & (kg)       & (kg)         & (kph) & (m/s) & (kph) & (m/s) & (kph) & (m/s) \\
\midrule
LS-3 & 383 & 121 & 115.03 & -0.86 & 174.04 & -1.76 & 212.72 & -3.4 \\
LS-4a & 361 & 121 & 114.87 & -0.8 & 172.3 & -2.33 & 210.59 & -4.5 \\
LS7wl & 350 & 150 & 103.77 & -0.73 & 155.65 & -1.47 & 180.00 & -2.66 \\
Nimbus 2 (20.3m) & 493 & 159 & 119.83 & -0.75 & 179.75 & -2.14 & 219.69 & -3.8 \\
Nimbus 3DM (24.6m PAS) & 820 & 168 & 114.97 & -0.57 & 157.42 & -0.98 & 222.24 & -2.3 \\
Nimbus 3D (24.6m PAS) & 712 & 168 & 93.64 & -0.46 & 175.42 & -1.48 & 218.69 & -2.5 \\
Nimbus 3D (24.6m PIL) & 621 & 168 & 87.47 & -0.43 & 163.86 & -1.38 & 204.27 & -2.3 \\
Nimbus 3 (24.6m) & 527 & 159 & 116.18 & -0.67 & 174.28 & -1.81 & 232.37 & -3.8 \\
Nimbus 3T & 577 & 310 & 141.7 & -0.99 & 182.35 & -1.89 & 243.13 & -4.0 \\
Nimbus 4DM (26m PAS) & 820 & 168 & 100.01 & -0.48 & 150.01 & -0.87 & 190.76 & -1.6 \\
Nimbus 4DM (26m PIL) & 729 & 168 & 94.31 & -0.46 & 141.47 & -0.82 & 179.9 & -1.5 \\
Nimbus 4D PAS & 743 & 303 & 107.5 & -0.5 & 142.74 & -0.83 & 181.51 & -1.6 \\
Nimbus 4D PIL & 652 & 303 & 99 & -0.46 & 133.73 & -0.78 & 170.07 & -1.5 \\
Nimbus 4 (26.4m) & 597 & 303 & 85.1 & -0.41 & 127.98 & -0.75 & 162.74 & -1.4 \\
PIK-20B & 354 & 144 & 102.5 & -0.69 & 157.76 & -1.59 & 216.91 & -3.6 \\
PIK-20D & 348 & 144 & 100 & -0.69 & 156.54 & -1.78 & 215.24 & -4.2 \\
PIK-20E & 437 & 80 & 109.61 & -0.83 & 166.68 & -2 & 241.15 & -4.7 \\
PIK-30M & 460 & 0 & 123.6 & -0.78 & 152.04 & -1.12 & 200.22 & -2.2 \\
PW-5 Smyk & 300 & 0 & 99.5 & -0.95 & 158.48 & -2.85 & 198.1 & -5.1 \\
Russia AC-4 (12.6m) & 250 & 0 & 99.3 & -0.92 & 140.01 & -1.8 & 170.01 & -2.9 \\
Stemme S-10 PAS & 850 & 0 & 133.47 & -0.83 & 167.75 & -1.41 & 205.03 & -2.3 \\
Stemme S-10 PIL & 759 & 0 & 125.8 & -0.82 & 158.51 & -1.33 & 193.74 & -2.2 \\
SZD-55-1 & 336 & 201 & 98.3 & -0.67 & 176.76 & -2.27 & 216.04 & -4.3 \\
Ventus A/B (16.6m) & 358 & 151 & 100.17 & -0.64 & 159.69 & -1.47 & 239.54 & -4.3 \\
Ventus B (15m) & 341 & 151 & 97.69 & -0.68 & 156.3 & -1.46 & 234.45 & -3.9 \\
Ventus 2C (18m) & 385 & 180 & 80.0 & -0.5 & 120.0 & -0.73 & 180.0 & -2.0 \\
Ventus 2Cx (18m) & 385 & 215 & 80.0 & -0.5 & 120.0 & -0.73 & 180.0 & -2.0 \\
Zuni II & 358 & 182 & 110 & -0.88 & 167 & -2.21 & 203.72 & -3.6 \\
Speed Astir & 351 &  90 &  90 & -0.63 & 105 & -0.72 & 157 & -2.0 \\
LS-6-18W &   330 & 140 &  90 & -0.51 & 100 & -0.57 & 183 & -2.0 \\
LS-8-15 &    325 & 185 &  70 & -0.51 & 115 & -0.85 & 173 & -2.0 \\
LS-8-18 &    325 & 185 &  80 & -0.51 &  94 & -0.56 & 173 & -2.0 \\
ASH-26E &    435 &  90 &  90 & -0.51 &  96 & -0.53 & 185 & -2.0 \\
ASG29-18 &   355 & 225 &  85 & -0.47 &  90 & -0.48 & 185 & -2.0 \\
ASW28-18 &   345 & 190 &  65 & -0.47 & 107 & -0.67 & 165 & -2.0 \\
\bottomrule
\end{longtable}
\end{small}
\end{maxipage}

\section{Profiles}

Profile files (extension \verb|.prf|) can be used to store
configuration settings used by XCSoar.  The format is a simple text
file containing \verb|<label>=<value>| pairs.  Certain values are text
strings delimited by double quotes, for example:
\begin{verbatim}
PilotName="Baron Richtoffen"
\end{verbatim}
All other values are numeric, including ones that represent boolean
values (true$=1$, false$=0$), for example:
\begin{verbatim}
StartDistance=1000
\end{verbatim}

All values that have physical dimensions are expressed in SI units
(meters, meters/second, seconds etc).

When a profile file is saved, it contains all configuration settings.
Profile files may be edited with a text editor to produce a smaller
set of configuration settings that can be given to other pilots to
load.  

When a profile file is loaded, only the settings present in that file
overwrite the configuration settings in XCSoar; all other settings are
unaffected.

The default profile file is generated automatically when configuration
settings are changed or when the program exits; this has the
file name \verb|xcsoar-registry.prf|.

The easiest way to create a new profile is to copy a previous one,
such as the default profile.  Copy the file, give it a logical name,
and then when XCSoar starts next time the new profile can be selected
and customised through the configuration settings dialogs.


\section{Checklist}

The checklist file (\verb|xcsoar-checklist.txt|) uses a similar format to
the airfield details file.  Each page in the checklist is preceded by
the name of the list in square brackets.  Multiple pages can be
defined (up to 20).

An example (extract):
\begin{verbatim}
[Preflight]
Controls
Harness, secure objects
Airbrakes and flaps
Outside
Trim and ballast
Instruments
Canopy
[Derigging]
Remove tape from wings and tail
...
\end{verbatim}

%%%%%%%%%%%%%% advanced stuff below..
\section{Tasks}
Task files (extension \verb|.tsk|) are currently in a special binary
format and cannot be easily edited other than in XCSoar or XCSoarPC.
However they are transferable between devices.

Work is under way to produce a text format that will make it easier
for users to edit the files or to export/import them for use with other
programs.

\section{Flight logs}

The software flight logger generates IGC files (extension \verb|.igc|)
according to the long naming convention described in the FAI 
document {\em Technical Specification for IGC-Approved GNSS Flight Recorders}.  
These files can be imported into other programs for analysis after flight.

\todonum[inline]{Describe use of FPLT to add events to the replay.}
The flight logs replay facility allows the files to include 
embedded commands to control XCSoar as if the user was interacting
with the program.  It does this by defining a special use for the
general-purpose `pilot event' IGC sentence:
\begin{quote}
LPLT event=StatusMessage Hello everybody
\end{quote}
This command will bring up a status message with the text ``Hello everybody''
when the line is reached during replay.

A future version of the flight logger will allow all input events to
be stored in the IGC file in this fashion, thus replaying the flight
will give a very faithful reproduction of the actual flight and how
the software was used.  This is expected to be useful for training and
coaching purposes.

The internal software logger has adjustable time steps, separate for
cruise and circling modes, via parameters in the configuration
settings.  Typically the circling time step is set to a smaller value
than cruise in order to give good quality flight logs for replay
purposes.

\section{Input events}

The input event file (extension \verb|.xci|) is a plain text file
designed to control the input and events in your glide computer.

You do not require access to the source code or understanding of
programming to write your own input event files but you do require
some advanced understanding of XCSoar and of gliding.

Some reasons why you might like to use xci:
\begin{itemize}
\item Modify the layout of button labels
\item Support a new set or layout of buttons (organiser hardware buttons)
\item Support an external device such as a bluetooth keyboard or gamepad
\item Customise any button/key event
\item Do multiple events from one key or glide computer triggered process
\end{itemize}
For more information on editing or writing or your own input event
file, see the {\em XCSoar Advanced Configuration Manual}.

\section{Language}
The language file (extension \verb|.xcl|) is a plain text file
designed to provide translations between English and other languages,
for messages and text displayed by XCSoar.

The format is quite simple, it is a list of text lines that XCSoar uses,
followed by an equals sign and the translation, thus:
\begin{verbatim}
[English text]=[Translated text]
\end{verbatim}

An example is provided below:
\begin{verbatim}
Hello=Hallo
\end{verbatim}

Several language files are available from the XCSoar website.

Upon startup, if the language file ``default.xcl'' is present and no
language file is specified in the configuration settings, then this
file will be loaded.

\section{Status}\label{sec:status}

Status files are text of the form {\em label=value}, arranged in
blocks of text where each block corresponds to an individual status
message.  These are delimited by double spaces.  Each block can
contain the following fields:
\begin{description}
\item[key]  This is the text of the status message.
\item[sound] Location of a WAV audio file to play when the status
  message appears.  This is optional.
\item[delay] Duration in milliseconds the status message is
  to be displayed.  This is optional.
\item[hide] A boolean (yes/no) that dictates whether the message
 is to be hidden (that is, not displayed). 
\end{description} 

Example:
\begin{verbatim}
key=Simulation\r\nNothing is real!
sound=\My Documents\XCSoarData\Start_Real.wav
delay=1500

key=Task started
delay=1500
hide=yes
\end{verbatim}
% 

\section{FLARM Identification file}\label{sec:flarm-ident-file}

The FLARM identification file \verb|xcsoar-flarm.txt| defines a table
of aircraft registrations or pilot names against the ICAO IDs that are
optionally broadcast by FLARM equipped aircraft.  These names are
displayed on the map next to FLARM traffic symbols, for matching ICAO
IDs.

The format of this file is a list of entries, one for each aircraft,
of the form {\em icao id=name}, where {\em icao id} is the six-digit
hex value of the ICAO aircraft ID, and {\em name} is free text
(limited to 20 characters), describing the aircraft and/or pilot name.
Short names are preferred in order to reduce clutter on the map
display.

Example:
\begin{verbatim}
DD8F12=WUS
DA8B06=Chuck Yeager
\end{verbatim}

Currently this file is limited to a maximum of 200 entries.

\section{Dialog files}\label{sec:dialog-files}

These files describe the layout of dialogs, including font size,
button size, and other aspects of the layout of the widgets within
the dialogs.  The files are written in an XML data format.

Users may want to edit these files, or use replacements prepared by
others, in order to change the layout of dialogs to suit their
preferences.  In particular, it is possible to hide configuration
settings or other data fields that the user is not interested in.

Refer to the {\em XCSoar Advanced Configuration Guide} for more details.
