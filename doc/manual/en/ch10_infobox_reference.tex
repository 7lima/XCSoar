\chapter{{\InfoBox} Reference}\label{cha:infobox}
Infobox data types are grouped into logical categories.

All InfoBoxes display their data in user-specified units.  Where data
is invalid, the displayed value will be '---' and the contents are
greyed out.  This happens, for example, when no terrain data is found
or it is not in range for the Terrain Elevation infobox type.

In the following description of the infobox data types, the first
title is as it appears in the infobox configuration dialog box, the
second title is the label used in the infobox title.

\newcommand{\ibi}[3]{%
\jindent{
\begin{tabular}{r}
{\bf #1} \\
\infobox{{#2}} \\
\end{tabular}}{#3}
}

\section{Altitude} %0,1,20,33,70

\ibi{Height GPS}{H GPS}{This is the height above mean sea level reported by the
GPS. Touchscreen/PC only: in simulation mode, this value is adjustable with the
up/down arrow keys. The right/left arrow keys also cause the glider to turn.}
\ibi{Height AGL}{H AGL}{This is the navigation altitude minus the terrain height
obtained from the terrain file. The value is coloured red when the glider is
below the terrain safety clearance height.}
\ibi{Terrain Elevation}{H Gnd}{This is the elevation of the terrain above mean
sea level obtained from the terrain file at the current GPS location.}
\ibi{Pressure Altitude}{H Baro}{This is the barometric altitude obtained from a
GPS equipped with pressure sensor or a supported external intelligent vario.}
\ibi{QFE GPS}{QFE GPS}{This is the height above the home airfield calculated
by subtracting the airfield elevation from the altitude reported by the GPS.}
\ibi{Flight level}{Flight Level}{Pressure Altitude given as Flight Level. Only available if barometric altitude available and correct QNH set.}
\ibi{Barogram}{Barogram}{Trace of altitude during flight.}

\section{Aircraft state} %3,6,23,32,37,47,54

\ibi{Bearing}{Bearing}{True bearing of the next waypoint. For AAT tasks, this
is the true bearing to the target within the AAT sector.}
\ibi{Speed ground}{V Gnd}{Ground speed measured by the GPS. If this infobox is
active in simulation mode, pressing the up and down arrows adjusts the speed, 
left and right turns the glider.}
\ibi{Track}{Track}{Magnetic track reported by the GPS. Touchscreen/PC only: If
this infobox is active in simulation mode, pressing the up and down arrows
adjusts the track.}
\ibi{Airspeed IAS}{V IAS}{Indicated Airspeed reported by a supported external
intelligent vario.}
\ibi{G load}{G}{Magnitude of G loading reported by a supported external
intelligent vario. This value is negative for pitch-down manoeuvres.}
\ibi{Bearing Difference}{Brng D}{The difference between the glider's track
bearing, to the bearing of the next waypoint, or for AAT tasks, to the bearing
to the target within the AAT sector. GPS navigation is based on the track
bearing across the ground, and this track bearing may differ from the glider's
heading when there is wind present. Chevrons point to the direction the glider
needs to alter course to correct the bearing difference, that is, so that the
glider's course made good is pointing directly at the next waypoint.  This
bearing takes into account the curvature of the Earth.}
\ibi{Airspeed TAS}{V TAS}{True Airspeed reported by a supported external 
intelligent vario.}
\ibi{Attitude indicator}{Horizon}{Attitude indicator (artifical horizon) display calculated from flightpath, supplemented with acceleration and variometer data if available.}

\section{Glide ratio} % 4,5,19,66,38,53,71

\ibi{L/D instantaneous}{L/D Inst}{Instantaneous glide ratio, given by the ground
speed divided by the vertical speed (GPS speed) over the last 20 seconds. 
Negative values indicate climbing cruise. If the vertical speed is close to
zero, the displayed value is '---'.
If this infobox is active, pressing the enter cursor button brings up the
bugs and ballast dialog.}
\ibi{L/D cruise}{L/D Cru}{The distance from the top of the last thermal, 
divided by the altitude lost since the top of the last thermal. Negative values
indicate climbing cruise (height gain since leaving the last thermal). If the
vertical speed is close to zero, the displayed value is '---'.}
\ibi{Final L/D}{Fin L/D}{The required glide ratio to finish the task, given by
the distance to go divided by the height required to arrive at the safety 
arrival altitude. Negative values indicate a climb is necessary to finish. If
the height required is close to zero, the displayed value is '---'.}
\ibi{Final GR}{Fin GR}{Geometric gradient to the arrival height above the final
waypoint. This is not adjusted for total energy.} 
\ibi{Next L/D}{WP L/D}{The required glide ratio to reach the next waypoint,
given by the distance to next waypoint divided by the height required to arrive
at the safety arrival altitude.  Negative values indicate a climb is necessary
to reach the waypoint.  If the height required is close to zero, the displayed
value is '---'.}
\ibi{L/D vario}{L/D vario}{Instantaneous glide ratio, given by the indicated
airspeed divided by the total energy vertical speed, when connected to an
intelligent variometer.  Negative values indicate climbing cruise. If the total
energy vario speed is close to zero, the displayed value is '---'.}
\ibi{L/D Average}{L/D Avg}{The distance made in the configured period of time ,
divided by the altitude lost since then. Negative values are shown as 
\^{ }\^{ }\^{ } and indicate climbing cruise (height gain). Over 200 of LD the
value is shown as +++ . You can configure the period of averaging in the Special config menu.
Suggested values for this configuration are 60, 90 or 120: lower values will be closed to 
LD INST, and higher values will be closed to LD Cruise. Notice that the distance 
is NOT the straight line between your old and current position: it's exactly the 
distance you have made even in a zigzag glide. This value is not calculated while 
circling.}

\section{Variometer} % 2,7,8,9,21,22,63,24,44

\ibi{Thermal last 30 sec}{TC 30s}{A 30 second rolling average climb rate based
of the reported GPS altitude, or vario if available.}
\ibi{Last Thermal Average}{TL Avg}{Total altitude gain/loss in the last thermal
divided by the time spent circling.} 
\ibi{Last Thermal Gain}{TL Gain}{Total altitude gain/loss in the last thermal.}
\ibi{Last Thermal Time}{TL Time}{Time spent circling in the last thermal.}
\ibi{Thermal Average}{TC Avg}{Altitude gained/lost in the current thermal,
divided by time spent thermalling.}
\ibi{Thermal Gain}{TC Gain}{The altitude gained/lost in the current thermal.}
\ibi{Thermal All}{TC All}{Time-average climb rate in all thermals.}
\ibi{Vario }{Vario}{Instantaneous vertical speed, as reported by the GPS, or the
intelligent vario total energy vario value if connected to one.}
\ibi{Netto Vario}{Netto}{Instantaneous vertical speed of air-mass, equal to
vario value less the glider's estimated sink rate. Best used if airspeed,
accelerometers and vario are connected, otherwise calculations are based on GPS
measurements and wind estimates.}
\ibi{Vario trace}{Vario}{Trace of vertical speed, as reported by the GPS, or the intelligent vario total energy vario value if connected to one.}
\ibi{Netto vario trace}{Netto}{Trace of vertical speed of air-mass, equal to vario value less the glider's estimated sink rate.}
\ibi{Thermal circling trace}{TC Circling}{Trace of average climb rate each turn in circling, based of the reported GPS altitude, or vario if available.}
\ibi{Climb band}{Climb band}{Graph of average circling climb rate (horizontal axis) as a function of height (vertical axis).}

\section{Atmosphere} % 25,26,48,49,50

\ibi{Wind Speed}{Wind V}{Wind speed estimated by XCSoar. Touchscreen/PC only:
Manual adjustment is possible by pressing the up/down cursor keys to adjust 
magnitude and left/right cursor keys to adjust bearing when the infobox is
active. Pressing the enter cursor key saves the wind value as the initial value
when XCSoar next starts.}
\ibi{Wind Bearing}{Wind B}{Wind bearing estimated by XCSoar. Touchscreen/PC
only: Manual adjustment is possible by pressing the up/down cursor keys to
adjust bearing when the infobox is active.}
\ibi{Outside Air Temperature}{OAT}{Outside air temperature measured by a probe
if supported by a connected intelligent variometer.}
\ibi{Relative Humidity}{RelHum}{Relative humidity of the air in percent as
measured by a probe if supported by a connected intelligent variometer.}
\ibi{Forecast Temperature}{MaxTemp}{Forecast temperature of the ground at the
home airfield, used in estimating convection height and cloud base in
conjunction with outside air temperature and relative humidity probe. 
Touchscreen/PC only: Pressing the up/down cursor keys adjusts this forecast
temperature.}

\section{MacCready} % 10,34,35,43

\ibi{MacCready Setting}{MacCready}{The current MacCready setting. This infobox
also shows whether MacCready is manual or auto. Touchscreen/PC only: Also used
to adjust the MacCready Setting if the infobox is active, by using the up/down
cursor keys. Pressing the enter cursor key toggles Auto MacCready mode.}
\ibi{Speed MacCready}{V MC}{The MacCready speed-to-fly for optimal flight to the
next waypoint. In cruise flight mode, this speed-to-fly is calculated for
maintaining altitude. In final glide mode, this speed-to-fly is calculated for
descent.}
\ibi{Percentage climb}{\% Climb}{Percentage of time spent in climb mode. These
statistics are reset upon starting the task.}
\ibi{Speed Dolphin }{V Opt}{The instantaneous MacCready speed-to-fly, making use
of Netto vario calculations to determine dolphin cruise speed in the glider's
current bearing. In cruise flight mode, this speed-to-fly is calculated for
maintaining altitude. In final glide mode, this speed-to-fly is calculated for
descent. In climb mode, this switches to the speed for minimum sink at the
current load factor (if an accelerometer is connected). When Block mode speed to
fly is selected, this infobox displays the MacCready speed.}

\section{Navigation} % 11,12,76,13,15,16,17,59,61,18,27,28,29,30,31,51,52,60,73

\ibi{Next Distance}{WP Dist}{The distance to the currently selected waypoint.
For AAT tasks, this is the distance to the target within the AAT sector.}
\ibi{Next Altitude Difference}{WP AltD}{Arrival altitude at the next waypoint
relative to the safety arrival altitude.}
\ibi{Next Altitude Arrival}{WP AltA}{Absolute arrival altitude at the next waypoint 
in final glide.}
\ibi{Next Altitude Required}{WP AltR}{Altitude required to reach the next turn
point.}
\ibi{Final Altitude Difference}{Fin AltD}{Arrival altitude at the final task
turn point relative to the safety arrival altitude.}
\ibi{Final Altitude Required}{Fin AltR}{Altitude required to finish the task.}
\ibi{Speed Task Average}{V Task Av}{Average cross country speed while on
current task, compensated for altitude.}
\ibi{Speed task instantaneous}{V Task Ins}{Instantaneous cross country speed
while on current task, compensated for altitude.  Equivalent to instantaneous Pirker cross-country speed.}
\ibi{Speed task achieved}{V Tsk Ach}{Achieved cross country speed while on
current task, compensated for altitude.}
\ibi{Final distance}{Fin Dis}{Distance to finish around remaining turn points.}
\ibi{AA Time}{AA Time}{ Assigned Area Task time remaining.}
\ibi{AA Distance Max}{AA Dmax}{ Assigned Area Task maximum distance possible for
remainder of task.}
\ibi{AA Distance Min}{AA Dmin}{ Assigned Area Task minimum distance possible for
remainder of task.}
\ibi{AA Speed Max}{AA Vmax}{ Assigned Area Task average speed achievable if
flying maximum possible distance remaining in minimum AAT time.}
\ibi{AA Speed Min}{AA Vmin}{ Assigned Area Task average speed achievable if
flying minimum possible distance remaining in minimum AAT time.}
\ibi{AA Distance Tgt}{AA Dtgt}{Assigned Area Task distance around target points
for remainder of task.}
\ibi{AA Speed Tgt}{AA Vtgt}{Assigned Area Task average speed achievable around
target points remaining in minimum AAT time.}
\ibi{Distance home}{Home Dis}{Distance to the home waypoint (if defined).}
\ibi{Online Contest Distance}{OLC}{Instantaneous evaluation of the flown
distance according to the configured Online-Contest rule set.}
\ibi{Task progress}{Progress}{Clock-like display of distance remaining along task, showing achieved taskpoints.}

\section{Waypoint} % 14,36,39,40,41,42,45,46,64

\ibi{Next Waypoint}{Next}{The name of the currently selected turn point. When
this infobox is active, using the up/down cursor keys selects the next/previous
waypoint in the task. Touchscreen/PC only: Pressing the enter cursor key brings
up the waypoint details.}
\ibi{Time of flight}{Time flt}{Time elapsed since takeoff was detected.}
\ibi{Time local}{Time loc}{GPS time expressed in local time zone.}
\ibi{Time UTC}{Time UTC}{GPS time expressed in UTC.}
\ibi{Task Time To Go}{Fin ETE}{Estimated time required to complete task,
assuming performance of ideal MacCready cruise/climb cycle.}
\ibi{Next Time To Go}{WP ETE}{Estimated time required to reach next waypoint,
assuming performance of ideal MacCready cruise/climb cycle.}
\ibi{Task Arrival Time}{Fin ETA}{Estimated arrival local time at task
completion, assuming performance of ideal MacCready cruise/climb cycle.}
\ibi{Next Arrival Time}{WP ETA}{Estimated arrival local time at next waypoint,
assuming performance of ideal MacCready cruise/climb cycle.}
\ibi{Task Req. Total Height Trend}{RH Trend}{Trend (or neg. of the variation) of
the total required height to complete the task.}
\ibi{Next altitude arrival}{WP AltA}{Absolute arrival altitude at the next waypoint in final glide.}
\ibi{Time under max. start height}{Start Height}{The contiguous period the ship has been below the task start max. height.}
\ibi{Task time to go (gnd spd)}{Fin ETE VMG}{Estimated time required to complete task, assuming current ground speed is maintained.}
\ibi{Next time to go (gnd spd)}{WP ETE VMG}{Estimated time required to reach next waypoint, assuming current ground speed is maintained.}

\section{Team code} % 55,56,57,58

\ibi{Team code}{TeamCode}{The current Team code for this aircraft. Use this
to report to other team members.}
\ibi{Team bearing}{Tm Brng}{The bearing to the team aircraft location at the
last team code report.}
\ibi{Team bearing difference}{Team Bd}{The relative bearing to the team aircraft
location at the last reported team code.}
\ibi{Team range}{Team Dis}{The range to the team aircraft location at the last
reported team code.}

\section{Device status} % 65,75

\ibi{Battery voltage/percent}{Battery}{Displays percentage of device battery remaining
(where applicable) and status/voltage of external power supply.}
\ibi{CPU load}{CPU}{CPU load consumed by XCSoar averaged over 5 seconds.}
\ibi{Free RAM}{Free RAM}{Free RAM as reported by the operating system.}

\section{Alternates} % 67,68,69

\ibi{Alternate 1 name}{Altrn 1}{Displays name and bearing to the best alternate
landing location.}
\ibi{Alternate 2 name}{Altrn 2}{Displays name and bearing to the second alternate
landing location.}
\ibi{Alternate 1 GR}{Altrn1 GR}{Geometric gradient to the arrival height above
the best alternate. This is not adjusted for total energy.}
