\chapter{History and Development}\label{cha:history-development}

\section{Product history}

XCSoar started as a commercial product developed by Mike Roberts (UK),
where it enjoyed a successful share of the market for several years
and going though several releases, the last being {\bf Version~2}.
Personal reasons prevented him from being able to continue supporting
the product and so in late 2004 he announced the licensing of the
source code to the GNU public license, as XCSoar {\bf Version~3}.  A support
website on Yahoo Groups was set up and the open source project started
to gain interest and input by developers.

In March 2005 the program was substantially enhanced and this resulted 
in {\bf Version~4.0} being released.  By this time, coordination of
the various development efforts on the source code became difficult
and time-consuming, so it was decided to move the project to
SourceForge, whereby all the software work could be managed by a
concurrent version management system.

In July 2005, {\bf Version~4.2} was released which addressed some
compatibility issues that were experienced with certain PDA and GPS
hardware configurations.

In September 2005, {\bf Version~4.5} was released.  This contained
major enhancements to the user interface including the introduction
of the `input event' system and language translation files.

In April 2006, {\bf Version~4.7} was released to Altair customers.
This contained stability and performance enhancements as well as many
bug fixes; and a new method for handling dialogs based on XML files.

In September 2006, {\bf Version~5.0} was released on all platforms,
Altair, PC, PDA.  This version contains many improvements and new
features and is based on extensive testing in flight and in
simulation.

In September 2007, {\bf Version~5.1.2} was released on all platforms,
Altair, PC, PDA.  This version contains many improvements and new
features and is based on extensive testing in flight and in
simulation.  Major improvements include a new map file format
incorporating JPG2000 compression, online contest support, additional
supported devices, FLARM radar screen, and overall improved stability,
reliability and accuracy of task calculations.  Many feature requests
from users have been incorporated into this release.

In February 2008, {\bf Version~5.1.6} was released. This contained 
numerous bugfixes and user interface enhancements, notably expanding 
the RASP and AAT functionality.

In March 2009, {\bf Version~5.2.2} was released. As well as user
interface improvements, several major features were introduced:
IGC files were digitally signed for validation in online contests 
such as OLC.
Windows CE-based PNA navigation devices were supported for the
first time. 
FLARM was integrated with the map display with support for the
flarmnet ID database.
Developers could now easily compile XCSoar from Linux desktop
computers.

In August 2009, {\bf Version~5.2.4} was released with internal
fixes and enhancements.

In December 2010, {\bf Version~6.0} was released. Following an
extensive rewrite of much of the program, many stability and
performance enhancements were made and startup times were 
dramatically reduced. A great many features were added including 
a vastly expanded task engine and editor, AAT support, new FLARM
and thermal-assistant displays. Many new languages were introduced,
and new translations could now be easily generated by users.

Significantly, the rewrite allowed XCSoar to run on UNIX-like
systems and Android devices, and the use of modern compiling 
tools enhanced the performance of the program on the current
generation of devices.

In March 2011, {\bf Version~6.0.7b} was released, the first
release to officially support Android.

\section{Get involved}

The success of the project is the result of many kinds of
contributions.  You do not have to be a software developer to help.

In general, there are perhaps five major ways of contributing, other
than working on the software itself:
\begin{description}
\item[Give feedback]
Ideas, suggestions, bug reports, encouragement and
constructive criticism are all very welcome and helpful.
\item[Setup suggestions]
Because XCSoar is so configurable, we rely to some extent on users to
think about how they would like the program to be set up.  Selection
of infobox layouts, button menus and button assignments require some
design thought, and making these available to the developers and other
users will help us provide good default settings.
\item[Data integrity]
Airspace and waypoint files need to be kept up to date, and it often
takes people with local knowledge to do this.
\item[Promotion]  The more users the software has, the better
 the product will be.  As more people use the software and give
 feedback, bugs are found more easily and improvements can occur at a
 greater pace.  You can help here, for example, by showing the
 software to others and by conducting demonstration and training
 sessions in your club.
\item[Documentation]  Naturally, the manual is always outdated, and we
  need help with maintaining it.
\end{description}

\section{Open source philosophy}

There are several benefits to having software like XCSoar open source.

\begin{itemize}
\item Firstly, it is free so pilots can try out the software at no
 cost and decide if it is suitable for their needs; and pilots are free
 to copy the program onto whatever Pocket PC device, PC or EFIS they like
 without charge.
\item You have access to the source code so you are free to change
 the software or use pieces of it in new free programs.
\item Having the source code available on the Internet means that it is
 subject to wide scrutiny and therefore bugs are easily and quickly fixed.
\item A large group of developers are available to help in troubleshooting
 and quickly implement new features upon request.
\item Open source software under the GNU public license cannot at a later
 date be made closed-source; so by using this software you will not be
 locked in to unspecified software costs in the future.
\end{itemize}

The full terms of the licensing agreement for XCSoar is given in
Appendix~\ref{cha:gnu-general-public}.

The development of XCSoar since its open source release has been
entirely a volunteer effort.  This does not preclude individual
developers or organisations from offering commercial support services.
The spirit of the project however suggests that in such cases the
commercial services are encouraged to produce some flow-on benefit
back to the wider community of users.

\section{Development process}

We try to incorporate new features as quickly as possible.  This has
to be balanced by the needs to not change substantially the interface
without appropriate warnings so users that upgrade do not get a shock.
This means that when we introduced the new button menu system in
version 4.5, it was necessary to also distribute a file that allowed
users to have the buttons assigned to their `legacy' function.

XCSoar, being used in flight, is a special kind of software because it
can be regarded as `mission-critical', and is a real-time system.
This has placed a very high emphasis on developers to perform a great
deal of testing before releasing changes to the public.  

Flight testing is certainly the best sort of test, but we have also
been able to conduct the bulk of testing by using XCSoar in a car,
and more recently, by replaying IGC flight logs.

In general, we don't want the program to crash or hang, ever, and if
it does so during testing, then whatever bug caused the problem has to
be fixed as top priority.

The software developers all keep in contact with each other through
the SourceForge developer's mailing list
\begin{quote}
\url{xcsoar-devel@lists.sourceforge.net}
\end{quote}
We try to coordinate our activities to avoid conflict and duplicated
effort, and to work together as a team.  If you would like to get
involved in the software development, send the developers an email.

\section{User base}

Who is using XCSoar?  Good question, and hard to answer.  Since no-one
pays for the product --- most people download the program anonymously
--- it is hard for anyone to keep track of how many users are out
there.

Statistics from the main website indicate there has been an average of
approximately twenty downloads per day between June 2005 and June
2006, and eighty downloads per day between June 2006 and September
2007.  Looking at how many people download the terrain and topography
data packs from the website indicates that it is used in many
countries and in nearly every continent.

XCSoar is used by a wide cross section of pilots, including early
post-solo through to experienced competition pilots.  There are many
`armchair' pilots who use XCSoar with gliding simulators, such as
Condor.

\section{Credits}\label{sec:credits}

Software developers:
\begin{itemize}
\item Santiago Berca \url{santiberca@yahoo.com.ar}
\item Tobias Bieniek \url{tobias.bieniek@gmx.de}
\item Robin Birch \url{robinb@ruffnready.co.uk}
\item Damiano Bortolato \url{damiano@damib.net}
\item Rob Dunning \url{rob@raspberryridgesheepfarm.com}
\item Samuel Gisiger \url{samuel.gisiger@triadis.ch}
\item Jeff Goodenough \url{jeff@enborne.f2s.com}
\item Lars H \url{lars_hn@hotmail.com}
\item Alastair Harrison \url{aharrison@magic.force9.co.uk}
\item Olaf Hartmann \url{olaf.hartmann@s1998.tu-chemnitz.de}
\item Mirek Jezek \url{mjezek@ipplc.cz}
\item Max Kellermann \url{max@duempel.org}
\item Russell King \url{rmk@arm.linux.org.uk}
\item Gabor Liptak \url{liptakgabor@freemail.hu}
\item Tobias Lohner \url{tobias@lohner-net.de}
\item Christophe Mutricy \url{xtophe@chewa.net}
\item Scott Penrose \url{scottp@dd.com.au}
\item Andreas Pfaller \url{pfaller@gmail.com}
\item Mateusz Pusz \url{mateusz.pusz@gmail.com}
\item Florian Reuter \url{flo.reuter@web.de}
\item Mike Roberts 
\item Matthew Scutter \url{yellowplantain@gmail.com}
\item Winfried Simon \url{winfried.simon@googlemail.com}
\item Google Inc., incl. Tom Stepleton \url{stepleton@google.com}
\item Simon Taylor \url{simon.taylor.uk@gmail.com}
\item Matthew Turnbull \url{matthewt@talk21.com}
\item Paolo Ventafridda \url{coolwind@email.it}
\item James Ward \url{jamesward22@gmail.com}
\item John Wharington \url{jwharington@gmail.com}


\end{itemize}


Documentation:
\begin{itemize}
\item Daniel Audier \url{osteocool@yahoo.fr}
\item Monika Brinkert \url{moni@sunpig.de}
\item Kevin Ford \url{ford@math.uiuc.edu}
\item Claus-W. Häbel \url{c-wh@online.de}
\item Stefan Murry \url{smurry@ao-inc.com}
\item Adrien Ott \url{adrien.ott@gmail.com}
\item Helmut J. Rohs \url{helmut.j.rohs@web.de}
\item Mauro H. M. Tamburini \url{maurotamburini@hotmail.com}
\item Wolfram Zirngibl \url{rueckwaertsflieger@wolframz.net}


\end{itemize}


Translators:
\begin{itemize}
\item Milan Havlik
\item*Zdenek Sebesta
\item Tobias Bieniek \url{tobias.bieniek@gmx.de}
\item Niklas Fischer \url{nf@nordthermik.de}
\item Peter Hanhart \url{peter.hanhart@schoensleben.ch}
\item Max Kellermann \url{max@duempel.org}
\item Helmut J. Rohs \url{helmut.j.rohs@web.de}
\item Philipp Wollschlegel \url{folken@kabelsalat.ch}
\item*Thomas Manousis
\item Miguel Valdiri Badillo \url{catastro1@tutopia.com}
\item Alexander Caldwell \url{alcald3000@yahoo.com}
\item Diego Guerrero \url{iccarod@hotmail.com}
\item*Hector Martin
\item Andres Miramontes \url{amiramon@gmail.com}
\item*Romaric Boucher
\item Sylvain Burger \url{sylvain.burger@wanadoo.fr}
\item*Dany Demarck
\item*Zoran Milicic
\item*Sasa Mihajlovic
\item Gabor Liptak \url{liptakgabor@freemail.hu}
\item*Kalman Rozsahegyi
\item*Enrico Girardi
\item*Lucas Marchesini
\item*Rick Boerma
\item Joop Gooden \url{joop.gooden@nlr.nl}
\item Hans van 't Spijker
\item Michal Jezierski \url{m.jezierski@finke.pl}
\item*Mateusz Pusz
\item Luke Szczepaniak \url{luke@silentflight.ca}
\item Mateusz Zakrzewski
\item*Tales Maschio
\item Luis Fernando Rigato Vasconcellos \url{fernando.rigato@gmail.com}
\item Monika Brinkert \url{moni@sunpig.de}
\item Nikolay Dikiy
\item Brtko Peter \url{p.brtko@facc.co.at}
\item Roman Stoklasa \url{rstoki@gmail.com}
\item*Aleksandar Cirkovic
\item*Patrick Pagden
\item 'zeugma'
\item Morten Jensen
\item Kostas Hellas \url{kostas.hellas@gmail.com}
\item Alexander Caldwell \url{alcald3000@yahoo.com}
\item Xavi Domingo \url{xavi@santmodest.net}
\item Arnaud Talon
\item Adrien Ott \url{adrien.ott@gmail.com}
\item Matthieu Gaulon
\item Filip Novkoski \url{f1novkoski@gmail.com}
\item Szombathelyi Zolt\'an \url{szombathelyi.zoltan@main.hu}
\item \'Ur Bal\'azs \url{urbalazs@gmail.com}
\item Paolo Pelloni \url{paolo@paolopelloni.it}
\item Piero Missa \url{pieromissa@virgilio.it}
\item Masahiro Mori \url{mron@n08.itscom.net}
\item Jinichi Nakazawa \url{jin-nakazawa@wkk.co.jp}
\item Mike Myungha Kuh
\item Rob Hazes
\item Thomas Amland \url{thomas.amland@gmail.com}
\item Quint Segers
\item Wil Crielaars \url{kawa1998@home.nl}
\item Krzysztof Kajda
\item Michał Tworek
\item Tiago Silva
\item Mario Souza
\item J\'ulio Cezar Santos Pires \url{juliocspires@gmail.com}
\item Wladimir Kummer de Paula
\item Pop Paul \url{poppali1@yahoo.com}
\item Dobrovolsky Ilya \url{ilya_42@inbox.ru}
\item Mats Larsson \url{mats.a.larsson@gmail.com}
%\item Мирзаева Асаль

\end{itemize}
*Contributions attributed to the LK8000 project (\url{http://www.lk8000.it/})


Other code and algorithms contributions come from:
\begin{description}
\item[Efêmeris] Jarmo Lammi
\item[Shapelib] Frank Warmerdam\\ \url{http://shapelib.maptools.org}
\item[Least squares] Curtis Olson\\ \url{http://www.flightgear.org/~curt}
\item[Formulário Aviônico] Ed Williams\\ \url{http://williams.best.vwh.net/avform.htm}
\item[JasPer] Michael D. Adams\\ \url{http://www.ece.uvic.ca/~mdadams/jasper/}
\item[Apoio ao Volkslogger] Garrecht Ingenieurgesellschaft
\item[Analisador de vento girando] Andr\'e Somers\\ \url{http://www.kflog.org/cumulus/}.


\end{description}

