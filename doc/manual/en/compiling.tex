\chapter{Compiling XCSoar}\label{cha:compiling}

The \texttt{make} command is used to launch the XCSoar build process.
You can learn more about the build system internals in chapter
\ref{cha:build}.

Most of this chapter describes how to build XCSoar on Linux, with
examples for Debian/Ubuntu.  A cross-compiler is used to build
binaries for other operating systems (for example Android and
Windows).

\section{Getting the Source Code}

The XCSoar source code is managed with
\href{http://git-scm.com/}{git}.  It can be downloaded with the
following command:

\begin{verbatim*}
git clone git://git.xcsoar.org/xcsoar/master/xcsoar.git
\end{verbatim*}

To update your repository, type:

\begin{verbatim*}
git pull
\end{verbatim*}

To update third-party libraries used by XCSoar (such as
\href{http://www.boost.org/}{Boost}), type:

\begin{verbatim*}
git submodule init
git submodule update
\end{verbatim*}

For more information, please read to the git documentation.

\section{Requirements}

The following is needed for all targets:

\begin{itemize}
\item GNU make
\item GNU compiler collection (\texttt{gcc}), version 4.6.2 or later
  or clang/LLVM 3.2 (with "make CLANG=y")
\item GNU gettext
\item \href{http://librsvg.sourceforge.net/)}{rsvg}
\item \href{http://www.imagemagick.org/}{ImageMagick 6.4}
\item \href{http://xmlsoft.org/XSLT/xsltproc2.html}{xsltproc}
\item Perl and XML::Parser
\end{itemize}

The following command installs these on Debian:

\begin{verbatim*}
apt-get install make \
  librsvg2-bin xsltproc \
  imagemagick gettext
\end{verbatim*}

\section{Target-specific Build Instructions}

\subsection{Compiling for Linux/UNIX}

The following additional packages are needed to build for Linux and
similar operating systems:

\begin{itemize}
\item \href{http://www.zlib.net/}{zlib}
\item \href{http://curl.haxx.se/}{CURL}
\item \href{http://www.libsdl.org/}{SDL}
\item \href{http://www.libsdl.org/projects/SDL\_ttf/}{SDL\_ttf}
\item \href{http://www.libpng.org/}{libpng}
\item \href{http://libjpeg.sourceforge.net/}{libjpeg}
\item OpenGL (Mesa)
\item to run XCSoar, you need one of the following fonts (Debian
  package): DejaVu (\texttt{fonts-dejavu}),
  Roboto (\texttt{fonts-roboto}),
  Droid (\texttt{fonts-droid}),
  Freefont (\texttt{fonts-freefont-ttf})
\end{itemize}

The following command installs these on Debian:

\begin{verbatim*}
apt-get install make g++ \
  zlib1g-dev \
  libsdl1.2-dev libfreetype6-dev \
  libpng-dev libjpeg-dev \
  libcurl4-openssl-dev \
  libxml-parser-perl \
  librsvg2-bin xsltproc \
  imagemagick gettext \
  fonts-dejavu
\end{verbatim*}

To compile, run:

\begin{verbatim*}
make
\end{verbatim*}

You may specify one of the following targets with \texttt{TARGET=x}:

\begin{tabular}{lp{8cm}}

\texttt{UNIX} & regular build (the default setting) \\

\texttt{UNIX32} & generate 32 bit binary \\

\texttt{UNIX64} & generate 64 bit binary \\

\texttt{OPT} & alias for UNIX with optimisation and no debugging \\

\end{tabular}

\subsection{Compiling for Android}

For Android, you need:

\begin{itemize}
\item \href{http://developer.android.com/sdk/}{Android SDK level 22}
\item \href{http://developer.android.com/sdk/ndk/}{Android NDK r10e}
\item \href{http://www.vorbis.com/}{Ogg Vorbis}
\item \href{http://ant.apache.org/}{Apache Ant}
\item {Java Native tools 
\begin{verbatim*}
sudo apt-get install gcj-native-helper ant vorbis-tools
\end{verbatim*}}
\end{itemize}

The \texttt{Makefile} assumes that the Android SDK is installed in
\verb|~/opt/android-sdk-linux| and the NDK is installed in
\verb|~/opt/android-ndk-r10e|.  You can use the options
\verb|ANDROID_SDK| and \verb|ANDROID_NDK| to override these paths.

After installing Java you will have to install at least one Android Platform SDK using the Android SDK Manager:

\begin{verbatim*}
~/opt/android-sdk-linux/tools/android
\end{verbatim*}

Load/update the IOIO source code:

\begin{verbatim*}
git submodule init
git submodule update
\end{verbatim*}

If you are using a 64-bit machine you might have a problem when you try to run the Android cross-compiler. 
The cross-compiler is a 32bit version of the compiler and the build script might complain that the necessary 
compiler is “not found”, which is obviously wrong since the file is there and accessible. Instead, the problem is
that the supporting 32-bit libraries are missing, which can be fixed through the following command:

\begin{verbatim*}
sudo apt-get install ia32-libs
\end{verbatim*}

To compile, run:

\begin{verbatim*}
make TARGET=ANDROID
\end{verbatim*}

Use one of the following targets:

\begin{tabular}{lp{8cm}}

\texttt{ANDROID} & for ARMv6 CPUs \\

\texttt{ANDROID7} & for ARMv7 CPUs \\

\texttt{ANDROID7NEON} & with
\href{http://www.arm.com/products/processors/technologies/neon.php}{NEON}
extension \\

\texttt{ANDROID86} & for x86 CPUs \\

\texttt{ANDROIDMIPS} & for MIPS CPUs \\

\texttt{ANDROIDFAT} & "fat" package for all supported CPUs \\

\end{tabular}

\subsection{Compiling for Windows}

To cross-compile to (desktop) Windows, you need the mingw-w64 version
of gcc:

 http://mingw-w64.sourceforge.net/

To compile, run one of the following:

\begin{verbatim*}
make TARGET=PC
\end{verbatim*}

Use one of the following targets:

\begin{tabular}{lp{8cm}}

\texttt{PC} & 32 bit Windows (i686) \\

\texttt{WIN64} & Windows x64 (amd64 / x86-64) \\

\texttt{WINE} & WineLib (experimental) \\

\texttt{CYGWIN} & Windows build with Cygwin (experimental) \\

\end{tabular}

\subsection{Compiling for Windows CE}

For PocketPC / Windows CE / Windows Mobile, you need mingw32ce:

\begin{itemize}
\item \href{http://max.kellermann.name/projects/cegcc/}{mingw32ce}
\end{itemize}

To compile, run:

\begin{verbatim*}
make TARGET=WM5X
\end{verbatim*}

Use one of the following targets:

\begin{tabular}{lp{8cm}}

\texttt{PPC2000} & PocketPC 2000 / Windows CE 3.0 \\

\texttt{PPC2003} & PocketPC 2003 / Windows CE 4.0 \\

\texttt{PPC2003X} & for XScale CPUs \\

\texttt{WM5} & Windows Mobile / Windows CE 5.0 \\

\texttt{WM5X} & for XScale CPUs \\

\texttt{ALTAIR} & for Triadis Altair \\

\end{tabular}

\subsection{Compiling for iOS and OS X}

For iOS and OS X, you can use the
\href{https://github.com/felixhaedicke/DarwinToolchainBuilder}{DarwinToolchainBuilder}
toolset. This toolset can be used regardless of whether compiling on OS X or on
any other operaing system. The installation instructions for these the tools
can be found in its
\href{https://github.com/felixhaedicke/DarwinToolchainBuilder/blob/master/README.md}{README.md}.

Furthermore, the following tools are required:
\begin{itemize}
\item png2icns from \href{http://icns.sourceforge.net}{libicns} to build for
  OS X
\item \href{https://alioth.debian.org/projects/dpkg}{dpkg} to build the iOS
  Cydia package
\item \href{http://cdrecord.org/private/cdrecord.html}{mkisofs} to build the
  OS X DMG package
\end{itemize}

To compile for iOS / AArch64, run:

\begin{verbatim*}
make TARGET=IOS64 cydia-deb
\end{verbatim*}

To compile for iOS / ARMv7, run:

\begin{verbatim*}
make TARGET=IOS32 cydia-deb
\end{verbatim*}

To compile for OS X / x86\_64, run:

\begin{verbatim*}
make TARGET=OSX64 dmg
\end{verbatim*}

To compile for OS X / x86, run:

\begin{verbatim*}
make TARGET=OSX32 dmg
\end{verbatim*}

The \texttt{Makefile} assumes that the Darwin toolchain is installed in
\verb|~/opt/darwin-toolchain| (only relevant if you are not compiling on OS X).
You can use the option \verb|DARWIN_TOOLCHAIN| to override this path.
To compile on Mac OS X, the directory where the library set (built with
DarwinToolchainBuilder) for the current target is installed needs to be
specified using the \verb|DARWIN_LIBS| option.

\subsection{Compiling for the Raspberry Pi}

You need an ARM toolchain.  For example, you can use the Debian
package \verb|g++-5-arm-linux-gnueabihf|:

\begin{verbatim*}
make TARGET=PI TCSUFFIX=-5
\end{verbatim*}

To optimize for the Raspberry Pi 2 (which has an ARMv7 with NEON
instead of an ARMv6):

\begin{verbatim*}
make TARGET=PI2 TCSUFFIX=-5
\end{verbatim*}

These targets are only used for cross-compiling on a (desktop)
computer.
If you compile on the Raspberry Pi, the default target will
auto-detect the Pi.

\subsection{Compiling for the Cubieboard}

To compile, run:

\begin{verbatim*}
make TARGET=CUBIE
\end{verbatim*}

This target is only used for cross-compiling on a (desktop) computer.
If you compile on the Cubieboard, the default target will auto-detect
the Cubieboard.

\subsection{Compiling for Kobo E-book Readers}

You need an ARM toolchain.  For example, you can use the Debian
package \verb|g++-5-arm-linux-gnueabihf|.

First build the third-party libraries:

\begin{verbatim*}
make TARGET=KOBO TCSUFFIX=-5 kobo-libs
\end{verbatim*}

To compile XCSoar, run:

\begin{verbatim*}
make TARGET=KOBO TCSUFFIX=-5
\end{verbatim*}

To build the kobo install file \texttt{KoboRoot.tgz}, you need the
following Debian packages:

\begin{verbatim*}
apt-get install fakeroot ttf-bitstream-vera
\end{verbatim*}

Then compile using this command:

\begin{verbatim*}
make TARGET=KOBO TCSUFFIX=-5 output/KOBO/KoboRoot.tgz
\end{verbatim*}

For this, you need the Debian package \verb|libc6-armhf-cross|.

\subsubsection{Building USB-OTG Kobo Kernel}

To build a USB-OTG capable kernel for the Kobo, clone the git
repository:

\begin{verbatim*}
git clone --branch kobo \
  git://git.xcsoar.org/xcsoar/max/linux.git
\end{verbatim*}

Configure the kernel using the file \texttt{kobo/kernel/otg.config}
from XCSoar's \texttt{git} repository.  Install a
\href{http://openlinux.amlogic.com:8000/download/ARM/gnutools/arm-2010q1-202-arm-none-linux-gnueabi-i686-pc-linux-gnu.tar.bz2}{gcc
  4.4 cross compiler}, for example in \texttt{/opt}.

Now type:

\begin{verbatim*}
make CROSS_COMPILE=/opt/arm-2010q1/bin/arm-none-linux-gnueabi- \
  ARCH=arm uImage
\end{verbatim*}

Copy \texttt{uImage} to the Kobo.  Kernel images can be installed with
the following command:

\begin{verbatim*}
dd if=/path/to/uImage of=/dev/mmcblk0 bs=512 seek=2048
\end{verbatim*}

Note that XCSoar's \texttt{rcS} script may overwrite the kernel image
automatically under certain conditions.  To use a new kernel
permanently, install it in \texttt{/opt/xcsoar/lib/kernel}.  Read the
file \texttt{kobo/rcS} to find out more about this.

To include kernel images in \texttt{KoboRoot.tgz}, copy
\texttt{uImage.otg} and \texttt{uImage.kobo} to
\texttt{/opt/kobo/kernel}.

\subsection{Editing the Manuals}

The XCSoar documententation, including the Developer Manual that you are 
reading right now, is written using the TeX markup language. You can edit
the source files with any text editor, although a specific TeX editor (e.g. LateXila) 
makes it easier. 

Source files are located in the en, fr, de, pl subdirectories of the doc/manual directory. 
The Developer manual is in the doc/manual/en directory. The generated files are put into
the output/manual directory.

To generate the PDF manuals, you need the TexLive package, plus some European languages.

The following command installs these on Debian:

\begin{verbatim*}
apt-get install texlive texlive-lang-french \
  texlive-lang-polish texlive-lang-german \
  texlive-extra-tools
\end{verbatim*}

The documentation is distributed as PDF files. Generating the PDFs from the 
TeX files is done by typing:

\begin{verbatim*}
make manual
\end{verbatim*}

A lot of warnings are generated... this is normal. Check for the presence of PDF files 
to ensure that the generation process was successful.
\section{Options}

\subsection{Parallel Build}

Most contemporary computers have multiple CPU cores.  To take
advantage of these, use the \texttt{make -j} option:

\begin{verbatim*}
make -j12
\end{verbatim*}

This command launches 12 compiler processes at the same time.

Rule of thumb: choose a number that is slightly larger than the number
of CPU cores in your computer.  12 is a good choice for a computer
with 8 CPU cores.

\subsection{Optimised Build}

By default, debugging is enabled and compiler optimisations are
disabled.  The resulting binaries are very slow.  During development,
that is helpful, because it catches more bugs.

To produce optimised binaries, use the option \texttt{DEBUG}:

\begin{verbatim*}
make DEBUG=n
\end{verbatim*}

Be sure to clean the output directory before you change the
\texttt{DEBUG} setting, because debug and non-debug output files are
not compatible.

The convenience target \texttt{OPT} is a shortcut for:

\begin{verbatim*}
TARGET=UNIX DEBUG=n TARGET_OUTPUT_DIR=output/OPT
\end{verbatim*}

It allows building both debug and non-debug incrementally, because two
different output directories are used.

\subsection{Compiling with ccache}

To speed up the compilation of XCSoar we can use \texttt{ccache} to cache the 
object files for us. All we have to do is install ccache and 
add 
\texttt{USE\_CCACHE=y} 
to the make command line:

\begin{verbatim*}
sudo apt-get install ccache
make TARGET=UNIX USE_CCACHE=y
\end{verbatim*}
